\documentclass[parskip=full]{scrartcl}
\usepackage[utf8]{inputenc} % use utf8 file encoding for TeX sources 
\usepackage[T1]{fontenc} % avoid garbled Unicode text in pdf 
\usepackage[german]{babel} % german hyphenation, quotes, etc 
\usepackage{hyperref} % detailed hyperlink/pdf configuration
\hypersetup{ % ‘texdoc hyperref‘ for options 
pdftitle={PSE Pflichtenheft}, %
bookmarks=true,%
}
\usepackage{csquotes} % provides \enquote{} macro for "quotes"
\usepackage{enumitem}


\title{PSE Pflichtenheft}
\author{Rukiye Devran, Tim Groß, Daniel Helmig, Orkhan Aliev, Florian Weber}
\begin{document}
	\begin{titlepage}
	\maketitle	
	\tableofcontents
	
	\end{titlepage}
\section{Einleitung}
\section{Zielbestimmung}
\subsection{Musskriterien}
\subsection{Wunschkritierien}
\section{Produkteinsatz}
	\subsection{Anwendungsbereich}
		
			Privatpersonen sollen in der Lage sein mit anderen Personen Schach zu spielen. Die Anwendung soll dies schnell und einfach ermöglichen.	
		
	\subsection{Zielgruppe}
		
			Die Anwendung richtet sich an Personen, die unterwegs eine Partie Schach spielen möchten.
		
	\subsection{Betriebsbedingungen}
		\begin{description}
			\item Die Anwendung soll täglich 24 Stunden verfügbar sein.
			\item Es sollen alle Versionen ab Android 4.4 unterstützt werden.
			\item Der Server soll Wartungsfrei laufen.	
		\end{description}
\section{Produktumgebung}
	\begin{description}
		\item Eine App für Mobilgeräte mit Android Betriebssystem.
		\item Ein Server zur Verwaltung von Partien und Spielsuche.		
	\end{description}
\section{Funktionelle Anforderungen}
\subsection{Client}
\subsection{Server}

\section{Produktdaten}

\subsection{Userdaten}
\begin{description}
	
\item[PD1010] Von jedem Nutzer ist ein eindeutiger Nutzername zu speichern.
\item[PD1020] Zusätzlich kann von jedem Nutzer die Email-Adresse, ein Passwort 	und weitere Daten von Authentifikations Providern gespeichert werden.
\item [PD1030] Für jeden Nutzer werden Statistiken über Siege/Niederlagen gespeichert.

\end{description}

\subsection{Spieldaten}
\begin{description}
	
\item[PD2010] Für jedes Spiel werden die Nutzernamen der beiden Spieler gespeichert.
\item[PD2020] Der Spielverlauf wird eindeutig abgespeichert.
\item[PD2030] Gegebenenfalls wird die Start Uhrzeit und die Dauer des Spiels gespeichert.

\end{description}

\section{Nichtfunktionale Anforderungen}
\begin{description}
	
	\item[NF10] Der Erstellungsprozess einer neuen Partie darf nicht länger als 20 Sekunden dauern.
	\item[NF20] Die Überprüfung auf Gültigkeit eines Zuges darf nicht länger als 2 Sekunden dauern.
	\item[NF30] Die Übermittlung einzelner Züge darf nicht länger als 2 Sekunden dauern.
	
\end{description}
\section{Globale Testfälle}
\section{Systemmodelle}
\subsection{Szenarien}
\begin{enumerate}
 
    \item
	\begin{description}
	\item[Zustand:] Die App ist geschlossen.
	\item[Aktion:] Die App wird erstmalig geöffnet.
	\item[Reaktion:] Es erscheint ein Fenster zur Eingabe eines Spielernamens.  \\	
	\end{description}
	
	\item
	\begin{description}
	\item[Zustand:] Die App ist geschlossen.
	\item[Aktion:] Die App wird zum wiederholten Mal geöffnet.
	\item[Reaktion:] Es erscheint das Hauptmenü und der Benutzer ist unter seinem Spielernamen eingeloggt. \\
	\end{description}
	
	\item
	\begin{description}
	\item[Zustand:] Die App befindet sich im Hauptmenü.
	\item[Aktion:] Der Benutzer klickt auf den Button "Statistiken".
	\item[Reaktion:] Es öffnet sich eine Accountübersicht mit allen gespeicherten Statistiken.  \\
	\end{description}
	
	\item
	\begin{description}
	\item[Zustand:] Die App befindet sich im Hauptmenü.
	\item[Aktion:] Der Benutzer klickt auf den Button "Spiel suchen".
	\item[Reaktion:] Der Benutzer kommt in eine Warteschlange für suchende Spieler.  \\
	\end{description}
	
	\item
	\begin{description}
	\item[Zustand:] Die App befindet sich im Hauptmenü.
	\item[Aktion:] Der Benutzer klickt auf den Button "Spieler suchen".
	\item[Reaktion:] Es erscheint eine Spielerübersicht mit Suchmöglichkeit.  \\
	\end{description}
	
	\item
	\begin{description}
	\item[Zustand:] Die App befindet sich im Hauptmenü.
	\item[Aktion:] Der Benutzer klickt auf den Button "Rangliste".
	\item[Reaktion:] Die Top 10 Spieler werden nach Elozahl sortiert aufgelistet.  \\
	\end{description}
	
	\item
	\begin{description}
	\item[Zustand:] Die App befindet sich im Spielersuchmenü.
	\item[Aktion:] Der Benutzer klickt auf einen anderen Spieler.
	\item[Reaktion:] Es erscheint eine Accountübersicht des Spielers mit Möglichkeit zur Herausforderung.  \\
	\end{description}
	
	\item 
	\begin{description}
	\item[Zustand:] Ein Spiel ist am laufen.
	\item[Aktion:] Der Benutzer klickt auf eine eigene Figur.
	\item[Reaktion:] Es werden alle möglichen Zugfelder markiert.  \\
	\end{description}
	
	\item 
	\begin{description}
	\item[Zustand:] Ein Spiel ist am laufen.
	\item[Aktion:] Der Benutzer klickt auf eine gegnerische Figur oder ein leeres Feld.
	\item[Reaktion:] Nichts passiert.  \\
	\end{description}
	
	
	\item
	\begin{description}
	\item[Zustand:] Eine Figur wurde angeklickt.
	\item[Aktion:] Der Benutzer klickt auf ein markiertes Feld.
	\item[Reaktion:] Die Figur wird gezogen und die Daten an den Server gesendet.  \\
	\end{description}
	
	\item
	\begin{description}
	\item[Zustand:] Ein Spiel ist am laufen.
	\item[Aktion:] Es wird ein Zug ausgeführt, welcher einen Spieler Matt oder Patt setzt.
	\item[Reaktion:] Es erscheint eine Benachrichtigung, die Elozahlen und Statistiken der Spieler werden aktualisiert, das Spiel wird beendet und abgespeichert.  \\
	\end{description}
	
	\item
	\begin{description}
	\item[Zustand:] Ein Spiel ist am laufen.
	\item[Aktion:] Ein Spieler klickt auf den Button "Aufgeben".
	\item[Reaktion:] Es erscheint eine Benachrichtigung, die Elozahlen und Statistiken der Spieler werden aktualisiert, das Spiel wird beendet und abgespeichert.  \\
	\end{description}
	
	\item
	\begin{description}
	\item[Zustand:] Ein Spiel ist am laufen.
	\item[Aktion:] Ein Spieler bietet Remis an.
	\item[Reaktion:] Der andere Spieler erhält eine Benachrichtigung mit Auswahlmöglichkeit.  \\
	\end{description}
	
	\item
	\begin{description}
	\item[Zustand:] Remis wurde angeboten.
	\item[Aktion:] Der Spieler klickt auf den Button "Annehmen".
	\item[Reaktion:] Es erscheint eine Benachrichtigung, die Elozahlen und Statistiken der Spieler werden aktualisiert, das Spiel wird beendet und abgespeichert.  \\
	\end{description}
	
	
	
	
\end{enumerate}
\subsection{Anwendungsfälle}

\end{document}
