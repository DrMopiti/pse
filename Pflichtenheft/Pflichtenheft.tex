\documentclass[parskip=full]{scrartcl}
\usepackage[utf8]{inputenc} % use utf8 file encoding for TeX sources 
\usepackage[T1]{fontenc} % avoid garbled Unicode text in pdf 
\usepackage[german]{babel} % german hyphenation, quotes, etc 
\usepackage{hyperref} % detailed hyperlink/pdf configuration
\hypersetup{ % ‘texdoc hyperref‘ for options 
pdftitle={PSE Pflichtenheft}, %
bookmarks=true,%
}
\usepackage{csquotes} % provides \enquote{} macro for "quotes"
\usepackage{enumitem}


\title{PSE Pflichtenheft}
\author{Rukiye Devran, Tim Groß, Daniel Helmig, Orkhan Aliev, Florian Weber}
\begin{document}
	\begin{titlepage}
	\maketitle	
	\tableofcontents
	
	\end{titlepage}
\section{Einleitung}
\section{Zielbestimmung}
Es soll ein Produkt namens .... als eine App angeboten werden mit der Personen gegeneinander Schach spielen können. Spieler sollen durch eine Spierlersuche andere Gegener finden und können sich in einer Rangliste vergleichen. 
\subsection{Musskriterien}
\begin{itemize}[nosep]
\item Alle Spieleregeln müssen implementiert werden.
\item Der Spieler muss gemäß den Spielreglen spielen.
\item Es muss eine Spielersuche geben.
\item Es muss eine GUI exestieren.
\item Es müssen die Schachfiguren angetippt und bewegt werden.
\end{itemize}
\subsection{Wunschkritierien}
\begin{itemize}[nosep]
\item Es soll einnen Account/Gastzugang geben.
\item Der Spieler kann ein Spiel nach dem beenden speichern.
\item Der Spieler kann eine Bedenkzeit einstellen.
\item Spieler könenn einen Chat mit dem entrpechenden Gegner führen.
\item Es soll ein Leaderboard oder ein Elosystem geben.
\item Die Anmeldung kann mit Facebook oder Google getätigt werden.
\item Es können zwei Spieler auf einem Gerät gleichzeitig spielen.
\item Es soll einen Revanche-Button geben.
\item Es soll verschiede Spielvarianten geben.
\end{itemize}
\subsection{Abgrenzkriterien}
\begin{itemize}[nosep]
\item Sprache: Java/Kotlin und XML
\item Umgebung: Android-Studio
\end{itemize}
\section{Produkteinsatz}
	\subsection{Anwendungsbereich}
		
			Privatpersonen sollen in der Lage sein mit anderen Personen Schach zu spielen. Die Anwendung soll dies schnell und einfach ermöglichen.	
		
	\subsection{Zielgruppe}
		
			Die Anwendung richtet sich an Personen, die unterwegs eine Partie Schach spielen möchten.
		
	\subsection{Betriebsbedingungen}
		\begin{description}
			\item Die Anwendung soll täglich 24 Stunden verfügbar sein.
			\item Es sollen alle Versionen ab Android 4.4 unterstützt werden.
			\item Der Server soll Wartungsfrei laufen.	
		\end{description}
\section{Produktumgebung}
	\begin{description}
		\item Eine App für Mobilgeräte mit Android Betriebssystem.
		\item Ein Server zur Verwaltung von Partien und Spielsuche.		
	\end{description}
	\section{Funktionelle Anforderungen}
\subsection{Benutzerfunktionen}
\begin{description}
	\item[F1010] \textbf{\textit{Registrieren: }} Ein beliebiger Android benutzer kann sich über die Start Seite des Applications registrieren lassen. Für die Registrierung im System sind folgende Angaben erforderlich: 
	\begin{itemize}
		\item eindeutige Benutzername
		\item gewünschte Passwort(muss mindestens 5 zeichen haben und davon 1 sonderzeichen)
		\item eigene eMail-Adresse
	\end{itemize}
	Der Registriervorgang wird nur dann erfolgreich abgeschlossen falls eMail-Adresse und der Benutzername im System jeweils eindeutig sind. Nach der erfolgreichen Registriervorgang bekommt der Benutzer per eMail seine Benutzername und Passwort.
	\item[F1020] \textbf{\textit{Anmelden: }} Nur nach dem bereits erfolgreichem Registrieren kann sich der Benutzer über die Start Seite anmelden. Dafür braucht der Nutzer:
	\begin{itemize}
		\item sein Benutzername
		\item sein Passwort
	\end{itemize}  
	\item[F1030] \textbf{\textit{Abmelden: }} Der Benutzer, der bereits angemeldet ist, kann sich wieder vom System abmelden.
	\item[F1040] \textbf{\textit{Gast: }} Der Benutzer, der sich nicht anmelden würde kann als Gastspieler den Spiel beitreten. Dafür muss er über die Start Seite Gast Knopfe drücken. Dann bekommt er vom System einen eindeutigen Benutzername. % muss noch füllen
	\item[F1050] \textbf{\textit{Passwort anfordern: }} Falls der registrierte Benutzer sein Passwort oder Benutzername vergessen hat ,so kann er die über die Start Seite anfordern. Dafür muss er in entsprechendem Feld seine eMail Adresse angeben. Dann bekommt er seiner Benutzername und seine Passwort per eMail automatisch zugeschickt.
	\item[F1060] \textbf{\textit{Passwort ändern: }}
	Der angemeldete Benutzer kann sein Passwort ändern. Dafür muss er sein aktuelles Passwort angeben und dann zweimal das neue Passwort, wobei sich diese Angaben nicht unterscheiden dürfen. Nach erfolgreiche Änderung des Passworts bekommt der Benutzer sein neues Pass per eMail.
	
	
\end{description}



\subsection{Initialisierung}
\begin{description}
	\item[F2010] Der Benutzer kann neue Spiele erzeugen ohne dabei einen anderen Spieler als Gegner angeben zu müssen. Ein anderen Spieler kann von diesem Spieler erzeugte Spiel unter dem Menüpunkt \textbf{Spiele} annehmen.
	\item[F2020] \textbf{\textit{Aufnahme eines Spieles: }} Der Benutzer kann schon eröffnete Spiele aufnehmen.
	\item[F2030] \textbf{\textit{Herausforderung: }} Der Benutzer kann unter Angabe eines gültigen Benutzernamens einen anderen Benutzer zum Spiel herausfordern oder nach dem ein Spiel zu Ende gekommen ist, können beide Spielern den anderen Spieler wieder auf dem Selben Bildschirm herausfordern.
	\item [F2040] \textbf{\textit{Akzeptieren einer Herausforderung: }} Der Benutzer kann Herausforderung zum Spiel \textbf{F0130} annehmen.
	\item [F2050] \textbf{\textit{Ablehnung einer Herausforderung: }} Der Benutzer kann Herausforderung zum Spiel \textbf{F0130} ablehnen.
	\item [F2060] \textbf{\textit{Statistiken: }} Jeder Benutzer, der im System angemeldet ist kann sich seine Statistiken :
	\begin{itemize}
		\item Wie viele Spiele gespielt wurden
		\item Wie viele mal gewonnen wurde
		\item Wie viele mal verloren wurde
		\item Wie viele Spiele Unentschieden geblieben sind
	\end{itemize}
	\item[F2070] \textbf{\textit{Freundschaftsanfragen senden: }}
	Ein bereits angemeldete Benutzer kann Freundschaftsanfragen senden indem er nur den Benutzername der jeweiligen Person angeben muss.
	\item[F2080] \textbf{\textit{Freundschaftsanfrage annehmen }} Ein bereits angemeldete Benutzer kann Freundschaftsanfragen annehmen.
	\item[F2090] \textbf{\textit{Freundschaftsanfrage ablehnen }} Ein bereits angemeldete Benutzer kann Freundschaftsanfragen ablehnen.
	
\end{description}

\subsection{Spielverlauf} 
\begin{description}
	\item[F3010]\textbf{\textit{Bewegungsmöglichkeiten: }}Der Benutzer kann während des Spiels falls er dran ist, einen Zug seiner Wahl unter Beibehaltung der Schach Spielregeln ziehen.
	\item[F3020] \textbf{\textit{Unentschieden(Remis) bieten: }} Der Benutzer, der gerade am Zug ist, kann gemäß der Spielregeln Unentschieden bieten.
	\item[F3030] \textbf{\textit{Unentschieden annehmen: }} Der Benutzer kann, wenn ihm Unentschieden angeboten wurde \textbf{F0220}, das Angebot akzeptieren.
	\item[F3040] \textbf{\textit{Unentschieden ablehnen: }} Der Benutzer kann, wenn ihm Unentschieden angeboten wurde \textbf{F0220}, das Angebot ablehnen.
	\item[F3050]\textbf{\textit{Nachrichtenaustausch: }} Die Benutzer können dazwischen Nachrichten austauschen.
\end{description}

\section{Produktdaten}

\subsection{Userdaten}
\begin{description}
	
\item[PD1010] Von jedem Nutzer ist ein eindeutiger Nutzername zu speichern.
\item[PD1020] Zusätzlich kann von jedem Nutzer die Email-Adresse, ein Passwort 	und weitere Daten von Authentifikations Providern gespeichert werden.
\item [PD1030] Für jeden Nutzer werden Statistiken über Siege/Niederlagen gespeichert.

\end{description}

\subsection{Spieldaten}
\begin{description}
	
\item[PD2010] Für jedes Spiel werden die Nutzernamen der beiden Spieler gespeichert.
\item[PD2020] Der Spielverlauf wird eindeutig abgespeichert.
\item[PD2030] Gegebenenfalls wird die Start Uhrzeit und die Dauer des Spiels gespeichert.

\end{description}

\section{Nichtfunktionale Anforderungen}
\begin{description}
	
	\item[NF10] Der Erstellungsprozess einer neuen Partie darf nicht länger als 20 Sekunden dauern.
	\item[NF20] Die Überprüfung auf Gültigkeit eines Zuges darf nicht länger als 2 Sekunden dauern.
	\item[NF30] Die Übermittlung einzelner Züge darf nicht länger als 2 Sekunden dauern.
	
\end{description}
	\section{Globale Testfälle}
\begin{description}
	\item[T1010] Ein Android Nutzer laden die Applikation und registriert sich im System \textbf{F1010}
	\item[T1020] Ein bereits registrierter Nutzer meldet sich mit seinen Benutzernamen und seiner Passwort in der Applikation an \textbf{F1020}
	\item[T1030] Ein bereits registrierter Nutzer meldet sich vom System ab \textbf{F1030}
	\item[T1040] Ein Gastspieler bekommt vom System einen eindeutigen Benutzername \textbf{F1040}
	\item[T1050] Ein bereits registrierter Benutzer fordert seine Passwort \textbf{F1050}
	\item[T1060] Ein bereits registrierter Benutzer ändert seine Passwort \textbf{F1060}
	\item[T2010] Der Benutzer erzeugt ein neues Spiel \textbf{F2010}
	\item[T2020] Der Benutzer führt ein schon eröffnetes Spiel fort \textbf{F2020}
	\item[T2030] Der Benutzer fordert unter Angabe eines gültigen Benutzernamens einen anderen Benutzer heraus \textbf{F2030}
	\item[T2040] Der Benutzer nimmt die Herausforderung an \textbf{F2040}
	\item[T2050] Der Benutzer lehnt die Herausforderung ab \textbf{F2050}
	\item[T2060] Der Benutzer schaut seine Statistiken an \textbf{F2060}
	\item[T2070] Der Benutzer sendet eine Freundschaftsanfrage \textbf{F2070}
	\item[T2080] Der Benutzer akzeptiert die Freundschaftsanfrage \textbf{F2080}
	\item[T2090] Der Benutzer lehnt die Freundschaftsanfrage ab \textbf{F2090}
	\item[T3010] Der Benutzer macht irgendeinen Zug seiner Wahl, wobei die Schachregeln eingehalten werden müssen \textbf{F3010}
	\item[T3020] Der Benutzer bietet ein Remis an, vorausgesetzt er ist am Zug \textbf{F3020}
	\item[T3030] Der Benutzer akzeptiert ein Remis \textbf{F3030}
	\item[T3040] Der Benutzer lehnt ein Remis ab \textbf{F3040}
	\item[T3050] Der Benutzer tauschen Nachrichten aus \textbf{F3050}
	
\end{description}
\section{Systemmodelle}
\subsection{Szenarien}
\begin{enumerate}
 
    \item
	\begin{description}
	\item[Zustand:] Die App ist geschlossen.
	\item[Aktion:] Die App wird erstmalig geöffnet.
	\item[Reaktion:] Es erscheint ein Fenster zur Eingabe eines Spielernamens.  \\	
	\end{description}
	
	\item
	\begin{description}
	\item[Zustand:] Die App ist geschlossen.
	\item[Aktion:] Die App wird zum wiederholten Mal geöffnet.
	\item[Reaktion:] Es erscheint das Hauptmenü und der Benutzer ist unter seinem Spielernamen eingeloggt. \\
	\end{description}
	
	\item
	\begin{description}
	\item[Zustand:] Die App befindet sich im Hauptmenü.
	\item[Aktion:] Der Benutzer klickt auf den Button \glqq Statistiken\grqq.
	\item[Reaktion:] Es öffnet sich eine Accountübersicht mit allen gespeicherten Statistiken.  \\
	\end{description}
	
	\item
	\begin{description}
	\item[Zustand:] Die App befindet sich im Hauptmenü.
	\item[Aktion:] Der Benutzer klickt auf den Button \glqq Spiel suchen\grqq.
	\item[Reaktion:] Der Benutzer kommt in eine Warteschlange für suchende Spieler.  \\
	\end{description}
	
	\item
	\begin{description}
	\item[Zustand:] Die App befindet sich im Hauptmenü.
	\item[Aktion:] Der Benutzer klickt auf den Button \glqq Spieler suchen\grqq.
	\item[Reaktion:] Es erscheint eine Spielerübersicht mit Suchmöglichkeit.  \\
	\end{description}
	
	\item
	\begin{description}
	\item[Zustand:] Die App befindet sich im Hauptmenü.
	\item[Aktion:] Der Benutzer klickt auf den Button \glqq Rangliste\grqq.
	\item[Reaktion:] Die Top 10 Spieler werden nach Elozahl sortiert aufgelistet.  \\
	\end{description}
	
	\item
	\begin{description}
	\item[Zustand:] Die App befindet sich im Spielersuchmenü.
	\item[Aktion:] Der Benutzer klickt auf einen anderen Spieler.
	\item[Reaktion:] Es erscheint eine Accountübersicht des Spielers mit Möglichkeit zur Herausforderung.  \\
	\end{description}
	
	\item 
	\begin{description}
	\item[Zustand:] Ein Spiel ist am laufen.
	\item[Aktion:] Der Benutzer klickt auf eine eigene Figur.
	\item[Reaktion:] Es werden alle möglichen Zugfelder markiert.  \\
	\end{description}
	
	\item 
	\begin{description}
	\item[Zustand:] Ein Spiel ist am laufen.
	\item[Aktion:] Der Benutzer klickt auf eine gegnerische Figur oder ein leeres Feld.
	\item[Reaktion:] Nichts passiert.  \\
	\end{description}
	
	
	\item
	\begin{description}
	\item[Zustand:] Eine Figur wurde angeklickt.
	\item[Aktion:] Der Benutzer klickt auf ein markiertes Feld.
	\item[Reaktion:] Die Figur wird gezogen und die Daten an den Server gesendet.  \\
	\end{description}
	
	\item
	\begin{description}
	\item[Zustand:] Ein Spiel ist am laufen.
	\item[Aktion:] Es wird ein Zug ausgeführt, welcher einen Spieler Matt oder Patt setzt.
	\item[Reaktion:] Es erscheint eine Benachrichtigung, die Elozahlen und Statistiken der Spieler werden aktualisiert, das Spiel wird beendet und abgespeichert.  \\
	\end{description}
	
	\item
	\begin{description}
	\item[Zustand:] Ein Spiel ist am laufen.
	\item[Aktion:] Ein Spieler klickt auf den Button \glqq Aufgeben\grqq.
	\item[Reaktion:] Es erscheint eine Benachrichtigung, die Elozahlen und Statistiken der Spieler werden aktualisiert, das Spiel wird beendet und abgespeichert.  \\
	\end{description}
	
	\item
	\begin{description}
	\item[Zustand:] Ein Spiel ist am laufen.
	\item[Aktion:] Ein Spieler bietet Remis an.
	\item[Reaktion:] Der andere Spieler erhält eine Benachrichtigung mit Auswahlmöglichkeit.  \\
	\end{description}
	
	\item
	\begin{description}
	\item[Zustand:] Remis wurde angeboten.
	\item[Aktion:] Der Spieler klickt auf den Button \glqq Annehmen\grqq.
	\item[Reaktion:] Es erscheint eine Benachrichtigung, die Elozahlen und Statistiken der Spieler werden aktualisiert, das Spiel wird beendet und abgespeichert.  \\
	\end{description}
	
	\item
	\begin{description}
	\item[Zustand:] Remis wurde angeboten.
	\item[Aktion:] Der Spieler klickt auf den Button \glqq Ablehnen\grqq.
	\item[Reaktion:] Die Benachrichtigung schließt sich, der andere Spieler erhält eine Mitteilung.  \\
	\end{description}
	
	
	
	
\end{enumerate}
\subsection{Anwendungsfälle}

\end{document}
