\documentclass[parskip=full]{scrartcl}
\usepackage[utf8]{inputenc} % use utf8 file encoding for TeX sources 
\usepackage[T1]{fontenc} % avoid garbled Unicode text in pdf 
\usepackage[german]{babel} % german hyphenation, quotes, etc 
\usepackage{hyperref} % detailed hyperlink/pdf configuration
\hypersetup{ % ‘texdoc hyperref‘ for options 
pdftitle={PSE Pflichtenheft}, %
bookmarks=true,%
}
\usepackage{csquotes} % provides \enquote{} macro for "quotes"
\usepackage{enumitem}


\title{PSE Pflichtenheft}
\author{Rukiye Devran, Tim Groß, Daniel Helmig, Orkhan Aliev, Florian Weber}
\begin{document}
	\begin{titlepage}
	\maketitle	
	\tableofcontents
	
	\end{titlepage}
\section{Einleitung}
\section{Zielbestimmung}
\subsection{Musskriterien}
\subsection{Wunschkritierien}
\section{Produkteinsatz}
	\subsection{Anwendungsbereich}
		\begin{description}
			Privatpersonen sollen in der Lage sein mit anderen Personen Schach zu spielen. Die Anwendung soll dies schnell und einfach ermöglichen.	
		\end{description}
	\subsection{Zielgruppe}
		\begin{description}
			Die Anwendung richtet sich an Personen, die unterwegs eine Partie Schach spielen möchten.
		\end{description}
	\subsection{Betriebsbedingungen}
		\begin{description}
			Die Anwendung soll täglich 24 Stunden verfügbar sein.
			Es sollen alle Versionen ab Android 4.4 unterstützt werden.
			Der Server soll Wartungsfrei laufen.	
		\end{description}
\section{Produktumgebung}
	\begin{description}
		Eine App für Mobilgeräte mit Android Betriebssystem.
		Ein Server zur Verwaltung von Partien und Spielsuche.		
	\end{description}
\section{Funktionelle Anforderungen}
\subsection{Client}
\subsection{Server}

\section{Produktdaten}

\subsection{Userdaten}
\begin{description}
	
\item[PD1010] Von jedem Nutzer ist ein eindeutiger Nutzername zu speichern.
\item[PD1020] Zusätzlich kann von jedem Nutzer die Email-Adresse, ein Passwort 	und weitere Daten von Authentifikations Providern gespeichert werden.
\item [PD1030] Für jeden Nutzer werden Statistiken über Siege/Niederlagen gespeichert.

\end{description}

\subsection{Spieldaten}
\begin{description}
	
\item[PD2010] Für jedes Spiel werden die Nutzernamen der beiden Spieler gespeichert.
\item[PD2020] Der Spielverlauf wird eindeutig abgespeichert.
\item[PD2030] Gegebenenfalls wird die Start Uhrzeit und die Dauer des Spiels gespeichert.

\end{description}

\section{Nichtfunktionale Anforderungen}
\section{Globale Testfälle}
\section{Systemmodelle}

\end{document}
