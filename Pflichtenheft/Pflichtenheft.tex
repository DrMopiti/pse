\documentclass[parskip=full]{scrartcl}
\usepackage[utf8]{inputenc} % use utf8 file encoding for TeX sources 

\usepackage[T1]{fontenc} % avoid garbled Unicode text in pdf 
\usepackage[german]{babel} % german hyphenation, quotes, etc 
\usepackage{hyperref} % detailed hyperlink/pdf configuration
\usepackage{graphicx}
\usepackage[toc]{glossaries}
\usepackage{caption}
\usepackage{pdfpages}
\hypersetup{ % ‘texdoc hyperref‘ for options 
pdftitle={PSE Pflichtenheft}, %
bookmarks=true,%
}
\usepackage{csquotes} % provides \enquote{} macro for "quotes"
\usepackage{enumitem}




\makeglossaries
\newglossaryentry{Schachregeln}
{
	name=Schachregeln,
	description={Schachregeln die gültige Züge und das Ende des Spiels definieren. \href{https://de.wikipedia.org/wiki/Schach\#Spielregeln}{https://de.wikipedia.org/wiki/Schach\#Spielregeln}}
}

\newglossaryentry{Schach}
{
	name=Schach,
	description={Das Brettspiel Schach}
}

\newglossaryentry{Schachfigur}
{
	name=Schachfigur,
	description={Die Spielsteine auf einem Schachbrett},
	plural=Schachfiguren
}

\newglossaryentry{Schachbrett}
{		
	name=Schachbrett,
	description={Das Spielbrett. Es ist quadratisch, zweifarbig und besteht aus 64 quadratischen Kacheln. Diese besitzen alle dieselbe Größe und sind in waagrechter und senkrechter Richtung abwechselnd eingefärbt},
	plural=Schachbretter
}

\newglossaryentry{Ausgangsposition} 
{
	name=Ausgangsposition,
	description={Vorgeschriebene Startposition aller Spielfiguren zu Beginn jeder Schachpartie}
}

\newglossaryentry{Remis}
{
	name=Remis,
	description={Der unentschiedene Ausgang einer Schachpartie, entweder durch Einigung beider Spieler oder durch Erreichen einer Stellung oder Zugfolge, welche ein Unentschieden erzwingt}
}

\newglossaryentry{Revanche}
{
	name=Revanche,
	description={Spiel, das dem Verlierer einer vorangegangene Partie die Chance gewährt, die Niederlage wieder wett zu machen, hier auch für den Gewinner möglich}
}

\newglossaryentry{Spieler}
{
	name={Spieler},
	description={Ein Nutzer der App},
}

\newglossaryentry{Elo}
{
	name={Elo},
	description={Wertung, die die Spielstärke beschreibt. \href{{https://de.wikipedia.org/wiki/Elo-Zahl}{https://de.wikipedia.org/wiki/Elo-Zahl}}}
}

\newglossaryentry{Schachmatt}
{
	name={Schachmatt},
	description={Der König steht im Schach und es gibt keinen regelkonformen Zug der das Schachgebot aufhebt. \href{https://de.wikipedia.org/wiki/Schachmatt}{https://de.wikipedia.org/wiki/Schachmatt}}
}

\newglossaryentry{Notation}
{
	name=Notation,
	description={Notation zur eindeutigen Dokumentation des Verlaufs einer Schachpartie. 	\href{https://de.wikipedia.org/wiki/Schachnotation}{https://de.wikipedia.org/wiki/Schachnotation}}
}

\newglossaryentry{Bedenkzeit} 
{
	name=Bedenkzeit,
	description={Maximale Zeit die einem Spieler für eine Schachpartie zur Verfügung steht. Es wird immer nur die Bedenkzeit des aktuell zu ziehenden Spielers dekrementiert. Bei Ablaufen seiner Bedenkzeit verliert der Spieler die Partie.}
}

\newglossaryentry{Android}
{
	name={Android},
	description={Ein Betriebssystem für mobile Geräte wie Smartphones, Tablets und auch Fernseher}
}
\newglossaryentry{Smartphone}
{
	name={Smartphone},
	description={Ein Mobiltelefon mit umfangreichen Computer-Funktionalitäten},
	plural = Smartphones
}
\newglossaryentry{Graphical User Interface}
{
	name={Graphical User Interface},
	description={Deutsch: Graphische Benutzeroberfläche}
}
\newglossaryentry{Schach-Engine}
{
	name={Schach-Engine},
	description={Ein Computer Programm zur Berechnung von möglichen Zügen \href{https://de.wikipedia.org/wiki/Schachprogramm}{https://de.wikipedia.org/wiki/Schachprogramm}}
}
\newglossaryentry{Patt}
{
	name={Patt},
	description={Patt im Schach tritt auf, falls dem am Zug befindlichen Spieler keine Zugmöglichkeit zur Verfügung steht, sein König sich jedoch nicht im Schach befindet}
}
\newglossaryentry{tote Stellung}
{
	name={tote Stellung},
	description={Eine tote Stellung liegt vor, falls eine Stellung entstanden ist, in welcher keiner der Spieler den gegnerischen König mit irgendeiner Folge von regelgemäßen Zügen mattsetzen kann. Zumeist hat hierbei keiner der beiden Spieler genug Figuren übrig, um den anderen Spieler mattsetzen zu können (z. B. König gegen König, König gegen König und Springer, König gegen König und Läufer). Es gibt auch Fälle, in denen etwa aufgrund einer verkeilten Bauernstruktur keiner der Spieler gewinnen kann.}
}

\newacronym{GUI}{GUI}{Graphical User Interface}



\begin{document}
	\begin{titlepage}
		
		
		%\centering
		%\vspace*{0.2\textheight}
		%{\Large PSE}\\[\baselineskip]
		%{\Huge PFLICHTENHEFT}\\[\baselineskip]\par
		%{\LARGE Rukiye Devran, Tim Groß, Daniel Helmig, Orkhan Aliev, Florian Weber}\par
	\includepdf{Titelblatt.pdf}
	\newpage	
	\tableofcontents
	\pagebreak
	
	\end{titlepage}
\section{Zielbestimmung}

\gls{Schach} ist weltweit bekannt und hat über viele Jahre hinweg eine tiefe kulturelle Bedeutung erlangt. Dadurch ist es zu einer Sportart geworden, die viele begeistert.
Deshalb soll im Rahmen der Praxis der Softwareentwicklung eine \gls{Schach}-App entwickelt werden, die es ermöglicht zu jeder Zeit Schach zu spielen.
Spieler sollen Gegner durch eine Spielsuche finden. Außerdem können sie sich in einer Rangliste vergleichen und es wird eine Statistik über jeden Spieler geführt.
Die App soll somit eine Partie \gls{Schach} ermöglichen, die man trotz weiter Entfernungen spielen kann, ohne sich gegenüber zu sitzen.
Letztendlich kann man durch häufiges Spielen bessere Spielfähigkeiten erlangen, sich mit anderen Spielern vergleichen und gegen seine Freunde spielen.

\subsection{Musskriterien}
\begin{description}
\item[KM1010] \textbf{\textit{\gls{Schachregeln}}}
\begin{itemize}
	\item Alle \gls{Schachregeln} sollen implementiert werden.
	\item Das Spielende soll definiert sein.	
\end{itemize}
\item[KM1020] \textbf{\textit{Einhaltung der \gls{Schachregeln}}}
\begin{itemize}
	\item Der Nutzer soll nur gültige Züge, die den \gls{Schachregeln} entsprechen, ausführen können.
\end{itemize}
\item[KM1030] \textbf{\textit{Spielsuche}}
\begin{itemize}
	\item Es soll dem Spieler möglich sein mit anderen Spielern zu spielen und zwar:
	\begin{itemize}
		\item Gegen einen zufälligen Gegner
		\item Gegen einen bestimmten Spieler, der mithilfe einer Suchfunktion bestimmt wird.
	\end{itemize}
\end{itemize}
\item[KM1040] \textbf{\textit{\gls{GUI}}}
\begin{itemize}
	\item Die Oberfläche besteht aus:
	\begin{itemize}		
		\item Login Seite: Hier soll sich der Nutzer mit Hilfe eines externen Kontos anmelden können.
		
		\item Hauptmenü: Hier soll der Nutzer dazu in der Lage sein, Spiele zu starten und seine Statistiken aufzurufen.
		
		\item Statistik Seite: Der Nutzer kann seine Spielstatistiken einsehen.
		
		\item Schachbrett: Hier kann der Nutzer in einer Partie seine Züge durchführen.
		
		\item Diverse Meldungen an den Spieler
	\end{itemize}
\end{itemize}
\item[KM1050] \textbf{\textit{\glspl{Schachfigur} bewegen}}
\begin{itemize}
	\item Es muss möglich sein, jede seiner Figuren anzutippen und damit auszuwählen.
	\item Für eine ausgewählte \gls{Schachfigur} müssen mögliche Züge angezeigt werden.
	\item Nur gültige Züge sollen ausgeführt werden können.
\end{itemize}
\end{description}

\subsection{Wunschkritierien}
\begin{description}

	\item[KW1010] \textbf{\textit{Account/Gastzugang}}
	\begin{itemize}
		\item Das Erstellen eines Accounts mit wählbarem und eindeutigen Benutzername soll 		möglich sein.
		\item Außerdem soll es die Möglichkeit geben, ohne Accounterstellung mittels eines 		Gastkontos zu spielen. Der Name wird automatisch vergeben.		
	\end{itemize}
	
	\item[KW1020] \textbf{\textit{Spiel speichern}}
	\begin{itemize}
		\item Beide Spieler sollen ein Spiel nach Beenden der Partie speichern können.
		\item Es soll eine Textdatei mit dem Partieverlauf in algebraischer \gls{Notation} erzeugt und auf dem Mobilgerät abgespeichert werden.
		\item Benutzer können die Textdatei öffnen und die Partie auf einem Brett nachspielen.
	\end{itemize}
	
	\item[KW1030] \textbf{\textit{Einstellbare Bedenkzeit und Spielfarbe}}
	\begin{itemize}
		\item Beide Spieler einer Partie sollen eine maximale \gls{Bedenkzeit} haben.
		\item Beim Herausfordern eines zuvor gewählten Spielers soll die \gls{Bedenkzeit} sowie die gewünschte Spielfarbe einstellbar sein.
		\item Zur Wahl der \gls{Bedenkzeit} soll eine Liste mit möglichen Zeiten erscheinen.
		\item Bei der Farbwahl soll zwischen Weiß, Schwarz und zufällig gewählt werden können.
	\end{itemize}
	
	\item[KW1040] 
	\begin{itemize} \textbf{\textit{Chat}}
		\item \gls{Spieler} können einen Chat mit dem entsprechenden Gegner führen.
		\item Dazu soll es während einer Partie ein Chatsymbol geben, welches ein Chatfenster öffnet.
		\item Dort können Nachrichten an den Gegner verschickt werden, sowie Spielmeldungen erscheinen.
	\end{itemize}
	
	\item[KW1050] \textbf{\textit{\gls{Elo}system und Rangliste}}
	\begin{itemize}
	\item Es soll für jeden Spieler eine \gls{Elo}zahl zur Messung der Spielstärke existieren.
	\item Nach jeder Partie sollen die Werte beider Spieler entsprechend einer Formel aktualisiert werden.
	\item Es soll eine Rangliste existieren, in welcher alle Spieler anhand ihrer \gls{Elo}zahl in absteigender Reihenfolge gelistet werden .
	\end{itemize}
	
	\item[KW1060] \textbf{\textit{Anmeldung mit Facebook oder Google}}
	\begin{itemize}
	\item Benutzer sollen die Möglichkeit haben, sich mit ihrem Google bzw. Facebook-Konto anzumelden.
	\item Der Account ist dann mit dem jeweiligen Konto verknüpft.
	\end{itemize}
	
	\item[KW1070] \textbf{\textit{Zwei Spieler auf einem Gerät}}
	\begin{itemize}
	\item Spieler sollen die Möglichkeit haben, zu zweit auf einem Gerät gegeneinander zu spielen.
	\item Die Symbole der schwarzen Figuren sollen zur besseren Benutzbarkeit auf den Kopf gedreht sein.
	\end{itemize}
	
	\item[KW1080] \textbf{\textit{\gls{Revanche}-Button}}
	\begin{itemize}
	\item Nach Beenden einer Partie sollen beide Spieler die Möglichkeit haben, einen Rückkampf zu fordern.
	\item Akzeptiert der Gegner die Herausforderung, so soll eine neue Partie gestartet werden.
	\item Die Spieler sollen bei der Revanche die jeweils andersfarbigen Spielfiguren zugeteilt bekommen wie bei der vorherigen Partie, die \gls{Bedenkzeit} soll dieselbe sein.
	\end{itemize}
	
\end{description}
\subsection{Abgrenzkriterien}
\begin{description}
\item[KA1010] \textbf{\textit{\gls{Schach-Engine}:}} Es soll keine selbst spielende \gls{Schach-Engine} implementiert werden.
\item[KA1020] \textbf{\textit{Zurücknahme:}} Die Spieler können Züge nicht zurücknehmen.
\item[KA1030] \textbf{\textit{Spielmodi:}} Bei der Spielsuche soll es keine Möglichkeit zur Modifikation des Spielmodus geben.
\item[KA1040] \textbf{\textit{Fremde Partien:}} Spiele von anderen \gls{Spieler}n können nicht live verfolgt werden.
\item[KA1050] \textbf{\textit{Partien einlesen:}} Es soll nicht möglich sein, andere Partien einzulesen.
\item[KA1060] \textbf{\textit{Mehrere Partien:}} Es ist nicht machbar, mehrere Partien gleichzeitig zu spielen.
\item[KA1070] \textbf{\textit{\gls{Bedenkzeit}:}} Der Spieler ist nicht dazu in der Lage, seinem Gegner mehr \gls{Bedenkzeit} zu gewähren.
\end{description}
\newpage
\section{Produkteinsatz}
	\subsection{Anwendungsbereich}
		
			Privatpersonen sollen in der Lage sein mit anderen Personen \gls{Schach} zu spielen. Die Anwendung soll dies schnell, einfach und mobil ermöglichen.	
		
	\subsection{Zielgruppe}
		
			Die Anwendung richtet sich an Personen mit einem Android \gls{Smartphone}, die unterwegs eine Partie \gls{Schach} spielen möchten.
		
	\subsection{Betriebsbedingungen}
		\begin{description}
			\item Die Anwendung soll täglich 24 Stunden verfügbar sein.
			\item Es sollen alle Versionen ab Android 4.4 unterstützt werden.
			\item Der Server soll wartungsfrei laufen.	
		\end{description}
	\newpage
\section{Produktumgebung}
	\subsection{Software}
		\begin{description}
			\item Eine App für Mobilgeräte mit \gls{Android} Betriebssystem ab Version 4.4.
			\item Ein Java Server zur Verwaltung von Partien und Spielsuche.		
		\end{description}
	\subsection{Hardware}
		\begin{description}			
			\item Ein Internetfähiges \gls{Smartphone} mit:
			\begin{itemize}
			\item \gls{Android} Betriebssystem
			\item Touchscreen
			\end{itemize}
		\item Ein virtueller Computer
		\end{description}
\pagebreak		  
\section{Funktionelle Anforderungen}
\subsection{Benutzerfunktionen}
\begin{description}
	\item[F1010] \textbf{\textit{Anmelden: }} Ein \gls{Android} Nutzer der auch einen Google Account besitzt, kann sich auf der Hauptseite der App erfolgreich anmelden. Für die Anmeldung im System sind folgende Angaben erforderlich:
	\begin{itemize}
		\item e-Mail Adresse, die mit Google Konto in Verbindung steht
		\item Google Konto Passwort 
	\end{itemize}  
	\item[F1020] \textbf{\textit{Gastzugang: }} Benutzer die keinen Google Account besitzen, können sich als Gast im System anmelden. Bei der Anmeldung wird ihnen ein eindeutiger Benutzername vom System zugewiesen. 
	\item[F1030] \textbf{\textit{Abmelden: }} Benutzer, die sich bereits mit ihrem Google Account angemeldet haben, können sich wieder vom System abmelden.
	
	\item [F1040] \textbf{\textit{Anzeige des eigenen, persönlichen Profils: }}  Der Benutzer kann sich sein persönliches Profil anzeigen lassen. Dabei sieht er seine Statistiken:
	
	\begin{itemize}
		\item Wie viele Spiele gespielt wurden
		\item Wie viele Male gewonnen wurde
		\item Wie viele Male verloren wurde
		\item Wie viele Spiele unentschieden gespielt wurden
		\item Elo-Zahl
		
	\end{itemize}
	\item[F1050] \textbf{\textit{Suche nach Benutzern: }} Der Benutzer kann mit der Suchfunktion nach anderen Benutzern des Systems anhand ihres Benutzernamens suchen.
	
	\item[F1060]  \textbf{\textit{Anzeigen der persönlichen Profile anderer Benutzer: }}
	Der Benutzer kann sich die persönlichen Profile von anderen Benutzern anzeigen lassen, wobei er den Benutzernamen und die Statistiken sieht. Die anderen Benutzer können sich genauso sein Profil anzeigen lassen.
	\item[F1070] \textbf{\textit{Chatten: }} Spieler können während des Spiels miteinander chatten.	
	
\end{description}

\subsection{Initialisierung}
\begin{description}
	\hypertarget{F2010}{\item[F2010]}\textbf{\textit{Eröffnung 	eines Spieles: }} Der Benutzer kann Spiele suchen, ohne dabei einen anderen Benutzer als Gegner angeben zu müssen. Dann bekommt er vom System einen Gegner zugewiesen, der auch genauso ein Spiel gesucht hat. Die Farbe ist zufällig und es steht eine feste \gls{Bedenkzeit} von 15 Minuten für jeden Spieler.
	
	\item[F2020] \textbf{\textit{Herausfordern: }} Nachdem ein entsprechender Gegner ausgesucht wurde \textbf{\textit{F1050}} kann der Benutzer ihn zum Spiel herausfordern.
	\item [F2030] \textbf{\textit{Akzeptieren einer Herausforderung: }} Der Benutzer kann die Herausforderung zum Spiel \textbf{F2030} annehmen.
	\item [F2040] \textbf{\textit{Ablehnung einer Herausforderung: }} Der Benutzer kann die Herausforderung zum Spiel \textbf{F2030} ablehnen.
	
	\item[F2050] \textbf{\textit{Spieleinstellungen bestimmen: }} Bei der Herausforderung einer bestimmten Person zum Spiel \textbf{\textit{F2020}} kann der Herausforderer folgende
	Spieleinstellungen aufstellen:
	\begin{enumerate}
		\item Zeit
		\item Farbe
	\end{enumerate}
	
	
	
\end{description}


\subsection{Spielverlauf} 
\begin{description}
	\item[F3010]\textbf{\textit{Schachfiguren bewegen: }}Der Benutzer kann während des Spiels, falls er dran ist, einen Zug seiner Wahl unter Beibehaltung der Schach Spielregeln ausführen wobei alle erlaubte Züge den Benutzer angezeigt werden.
	\item[F3020] \textbf{\textit{Unentschieden(Remis) bieten: }} Ein Spieler, der Remis anbieten möchte, tut dies, nachdem er einen Zug auf dem Schachbrett ausgeführt und nachdem seine Uhr angehalten und die des Gegners sich in Gang gesetzt hat.
	\item[F3030] \textbf{\textit{Remis annehmen: }} Der Benutzer kann, wenn ihm Unentschieden angeboten wurde \textbf{F3020}, das Angebot annehmen.
	
	\item[F3040] \textbf{\textit{Remis ablehnen: }} Der Benutzer kann, wenn ihm Unentschieden angeboten wurde \textbf{F3020}, das Angebot ablehnen.
	
	\item[F3050] \textbf{\textit{Aufgeben: }} Während des Spiels kann jeder Spieler jederzeit aufgeben.
	\item[F3060] \textbf{\textit{Spiel enden: }} Ein Spiel endet, falls einer von beiden Spielern den anderen Schachmatt gesetzt hat, oder eine von unten gezählten Remis Situation auftritt:
	\begin{enumerate}
		\item Falls einer von beiden Spielern Remis angeboten und der andere Spieler das Angebot angenommen hat
		\item Falls ein \gls{Patt} auftritt
		\item Falls eine \gls{tote Stellung} vorliegt
		\item Wenn eine identische Stellung mit gleichen Zugmöglichkeiten und demselben Spieler am Zug mindestens zum dritten Mal auf dem Schachbrett entstanden ist 
	\end{enumerate}
	\item[F3070] \textbf{\textit{Rückkampf anbieten: }} Nachdem ein Spiel zu Ende gekommen ist, muss jeder Spieler den anderen Spieler einen Rückkampf anbieten zu können.		
	\item[F3080] \textbf{\textit{Multiplayer auf einem Gerät: }}  Mit einer Multiplayer Funktion können zwei Spieler auf einem Gerät gegeneinander spielen.
\end{description}
\newpage
\section{Nichtfunktionale Anforderungen}
\begin{description}
	
	\item[NF1010] \textbf{\textit{Start der App: }} Das Starten der App soll auf aktuellen Geräten maximal 3 Sekunden dauern. 
	\item[NF1020] \textbf{\textit{Spiel Erstellung: }} Der Erstellungsprozess \hyperlink{F2010}{\textbf{F2010}} einer neuen Partie darf nicht länger als 5 Sekunden dauern.
	\item[NF1030] \textbf{\textit{Ermittlung von Spielzügen: }} Die Ermittlung \hyperlink{F3010}{\textbf{F3010}} an möglichen gültigen Zügen darf nicht länger als 0,1 Sekunden dauern.
	\item[NF1040] \textbf{\textit{Herausforderung: }} Falls ein Spieler herausgefordert wird, hat dieser zwei Minuten Zeit diese anzunehmen. Danach verfällt die Herausforderung. 
	\item[NF1050] \textbf{\textit{Weiterleitung von Nachrichten: }} Die Weiterleitung und Überprüfung einzelner Züge soll auf dem Server nicht länger als 2 Sekunden dauern.
	\item[NF1060] \textbf{\textit{Verlust von Paketen: }} Bei Übertragungen zwischen zwei Geräten sollen keine Pakete unbemerkt verloren gehen.
	\item[NF1070] \textbf{\textit{Manipulation von Nachrichten: }} Nachrichten/Spielzüge sollen unverändert am Empfänger eintreffen. Sollten Änderungen vorgenommen worden sein, soll dies vom Empfänger erkannt werden können.
	\item[NF1080] \textbf{\textit{Wartung: }} Der Server soll wartungsfrei und ohne Neustarts auskommen.

	
\end{description}

\newpage
\section{Produktdaten}

\subsection{System-Daten auf mobilen Geräten}
\begin{description}
	
	\item[PD1010] \textbf{\textit{Die Einstellungen}}
	\begin{description}
	\item Die Einstellungen beinhalten:
		

	\begin{itemize}
		\item Benutzername
		\item Verbindungsdaten vom Server
		\item Sonstige Einstellungen
	\end{itemize}
\end{description}


\end{description}



\begin{description}
	
	\item[PD1020] \textbf{\textit{Spieldaten von der aktuell laufender Partie}}
		\begin{description}
	\item Zu den Spieldaten gehört:
		\begin{itemize}
		\item Partiekennung
		\item Aktueller Zustand des Schachbretts	
	\end{itemize}

	
	\end{description}	
\end{description}


\subsection{System-Daten auf zentralem Server}
\begin{description}
	
\item[PD2010] \textbf{\textit{Server-Einstellungen}}
\item[PD2020] \textbf{\textit{Partie-Informationen}}
\begin{description} 
	\item Für jede Partie wird folgendes gespeichert:
	\begin{itemize}
		\item Partiekennung
		\item Benutzername beider \gls{Spieler}
		\item Aktueller Zustand des Schachbrettes
		\item Alle bisherigen Züge
	\end{itemize}

\end{description}
\end{description}

\subsection{Benutzer-Daten auf mobilen Geräten}
\begin{description}
\item[PD3010]  \textbf{\textit{Benutzerinformationen}}
\begin{description} 
	\item Folgende Daten werden über jeden Benutzer gespeichert:
	\begin{itemize}
		\item Benutzername
		\item Eigene Spielstatistiken
	\end{itemize}
	
\end{description}
\end{description}

\subsection{Benutzer-Daten auf zentralem Server}
\begin{description}
	\item[PD4010]  \textbf{\textit{Benutzerinformationen}}
	\begin{description} 
		\item Folgende Daten werden über jeden Benutzer gespeichert:
		\begin{itemize}
			\item Benutzername
			\item e-Mail-Adresse
			\item Sonstige Informationen von Login Providern
			\item Spielstatistiken
		\end{itemize}
		\item Zusätzlich wird eine globales Rangliste über die besten \gls{Spieler} gespeichert.
	\end{description}
\end{description}
\newpage


\begin{figure}[htp]
	\section{GUI Entwürfe}
	
	\begin{minipage}[t]{6cm}
		\vspace{0pt}
		%\centering
		\includegraphics[height=100mm]{gui_login.png}
		\caption{GUI Login}
		\label{fig:GUI Login}
	\end{minipage}
	\hfill
	\begin{minipage}[t]{6cm}
		\vspace{0pt}
		\begin{description}
			 \item[G1010] \textbf{\textit{Anmeldemenü: }} Startbildschirm der App. Hier kann man sich entweder über eine Google-Anmeldung anmelden oder als Gast spielen.
		\end{description}
	\end{minipage}

	\begin{minipage}[t]{6cm}
		\vspace{0pt}
		%\centering
		\includegraphics[height=100mm]{google_login.png}
		\caption{Google Login}
		\label{fig:Google Login}
	\end{minipage}
	\hfill
	\begin{minipage}[t]{6cm}
		\vspace{0pt}
		\begin{description}
			\item[G1020] \textbf{\textit{Google Login: }} Nach dem drücken auf den Knopf "Anmeldung mit Google" öffnet sich das Fenster für den Login.
		\end{description}
	\end{minipage}
\end{figure}

\begin{figure}[htp]
	\begin{minipage}[t]{6cm}
		\vspace{0pt}
		%\centering
		\includegraphics[height=100mm]{hauptmenu.png}
		\caption{Hauptmenü}
		\label{fig:Hauptmenü}
	\end{minipage}
	\hfill
	\begin{minipage}[t]{6cm}
		\vspace{0pt}
		\begin{description}
			\item[G1030] \textbf{\textit{Hauptmenü: }} Nach dem man sich angemeldet hat, erscheint das Hauptmenü mit den Buttons:
			Sofortspiel: Spiel gegen einen zufälligen Gegner.
			Spielersuche: Man kann einen Spieler mit seinem Spiernamen suchen und herausfordern.
			Rangliste/Statistik.
		\end{description}
	\end{minipage}

	\begin{minipage}[t]{6cm}
		\vspace{0pt}
		%\centering
		\includegraphics[height=100mm]{ingame.png}
		\caption{Spielbrett}
		\label{fig:Spielbrett}
	\end{minipage}
	\hfill
	\begin{minipage}[t]{6cm}
		\vspace{0pt}
		\begin{description}
			\item[G1040] \textbf{\textit{Spielbrett: }} Bei einem Spiel werden nach der Wahl einer Figur die möglichen Züge angezeigt. Außerdem gibt es zwei Buttons Unentschieden und Aufgeben.
		\end{description}
	\end{minipage}
\end{figure}

\begin{figure}[htp]
	\begin{minipage}[t]{6cm}
		\vspace{0pt}
		%\centering
		\includegraphics[height=100mm]{Sieg.png}
		\caption{Sieg}
		\label{fig:Sieg}
	\end{minipage}
	\hfill
	\begin{minipage}[t]{6cm}
		\vspace{0pt}
		\begin{description}
			\item[G1050] \textbf{\textit{Sieg: }} Wenn man eine Partie gewonnen hat erscheint ein Fenster mit der Meldung, dass man gewonnen hat und mit den zusätzlichen Knöpfen: Menü, Speichern und noch ein Spiel.
		\end{description}
	\end{minipage}

	\begin{minipage}[t]{6cm}
		\vspace{0pt}
		%\centering
		\includegraphics[height=100mm]{aufgeben.png}
		\caption{Aufgaben}
		\label{fig:Aufgeben}
	\end{minipage}
	\hfill
	\begin{minipage}[t]{6cm}
		\vspace{0pt}
		\begin{description}
			\item[G1060] \textbf{\textit{Aufgeben: }} Wenn man auf den Aufgeben Button drückt, erscheint ein Fenster, ob man wirklich aufhören möchte mit den Buttons: Ja und Nein.
		\end{description}
	\end{minipage}
\end{figure}

\begin{figure}[htp]
	\begin{minipage}[t]{6cm}
		\vspace{0pt}
		%\centering
		\includegraphics[height=100mm]{unentschieden.png}
		\caption{Unentschieden}
		\label{fig:Unentschieden}
	\end{minipage}
	\hfill
	\begin{minipage}[t]{6cm}
		\vspace{0pt}
		\begin{description}
			\item[G1070] \textbf{\textit{Unentschieden: }} Wenn man die Partie mit einem Unentschieden beendet, taucht ein Fenster auf. Man kann bei Bedarf noch einmal gegen den gleichen Spieler spielen.
		\end{description}
	\end{minipage}

	\begin{minipage}[t]{6cm}
		\vspace{0pt}
		%\centering
		\includegraphics[height=100mm]{spielersuche.png}
		\caption{Spielersuche}
		\label{fig:Spielersuche}
	\end{minipage}
	\hfill
	\begin{minipage}[t]{6cm}
		\vspace{0pt}
		\begin{description}
			\item[G1080] \textbf{\textit{Spielersuche: }} Wenn man auf den Button Spielersuche im Menü [G1030] drückt, kann man mit einer Suchleiste Freunde suchen und gegen sie spielen.
		\end{description}
	\end{minipage}
\end{figure}

\begin{figure}[htp]
	\begin{minipage}[t]{6cm}
		\vspace{0pt}
		%\centering
		\includegraphics[height=100mm]{spielerprofil.png}
		\caption{Freunde herausfordern}
		\label{fig:Freunde herausforern}
	\end{minipage}
	\hfill
	\begin{minipage}[t]{6cm}
		\vspace{0pt}
		\begin{description}
			\item[G1090] \textbf{\textit{Freunde Herausfordern: }} Nach dem man bei der Spielersuche einen Freund gefunden und angewählt hat, öffnet sich die Statistik zu diesem Freund. Danach kann man ihn herausfordern.
		\end{description}
	\end{minipage}

	\begin{minipage}[t]{6cm}
		\vspace{0pt}
		%\centering
		\includegraphics[height=100mm]{custom_match.png}
		\caption{Spieleinstellung}
		\label{fig:Spieleinstellung}
	\end{minipage}
	\hfill
	\begin{minipage}[t]{6cm}
		\vspace{0pt}
		\begin{description}
			\item[G1110]\textbf{\textit{Spieleinstellung: }} Nachdem man auf den Button Herausfordern gedrückt hat, erscheint ein Fenster mit der Einstellung für die Farben, weiß, schwarz oder zufällig, und die Einstellung für die Zeit in Minuten.
		\end{description}
	\end{minipage}
\end{figure}

\begin{figure}[htp]
	\begin{minipage}[t]{6cm}
		\vspace{0pt}
		%\centering
		\includegraphics[height=100mm]{leaderboard.png}
		\caption{Rangliste}
		\label{fig:Rangliste}
	\end{minipage}
	\hfill
	\begin{minipage}[t]{6cm}
		\vspace{0pt}
		\begin{description}
			\item[G1120] \textbf{\textit{Rangliste: }} Wenn man auf den Button Rangliste/Statistik im Menü [G1030] drückt, erscheint die Rangliste der Nutzer aufgelistet in einer Tabelle mit den jeweiligen Punkten.
		\end{description}
	\end{minipage}

	\begin{minipage}[t]{6cm}
		\vspace{0pt}
		%\centering
		\includegraphics[height=100mm]{statistik.png}
		\caption{Statistik}
		\label{fig:Statistik}
	\end{minipage}
	\hfill
	\begin{minipage}[t]{6cm}
		\vspace{0pt}
		\begin{description}
			\item[G1130] \textbf{\textit{Statistik: }} Beim weiteren Scrollen von [G1120] nach unten erscheint die Spielstatistik des Nutzers. Dort werden Angaben über die gewonnen Spiele, verlorenen Spiele, Gewinnrate, Unentschieden, Gesamtspielanzahl und dem Elo Wert gemacht.
		\end{description}
	\end{minipage}
\end{figure}
\newpage
\clearpage

\section{Globale Testfälle}
\begin{description}
	\item[T1010] \textbf{\textit{Anmelden: }} \\
	\begin{enumerate}
		\item 
		\begin{description}
			\item[Stand] App ist geschlossen
			\item[Aktion] Benutzer öffnet App 
			\item[Reaktion] Es öffnet sich die Startseite des App-s und Anmeldemöglichkeit mit Google Account  ist in der oberen Mitte
		\end{description} 
		\item 
		\begin{description}
			\item[Stand] Benutzer befindet sich auf der Hauptseite des App-s
			\item[Aktion] Benutzer drückt auf dem Knopf \textit{\textbf{Anmelden}} an und gibt seine Google Kennung  \textit{\textbf{mustermann@gmail.com }}als  eMail Adresse und \textit{\textbf{max1980!}} als Passwort ein.
			\item[Reaktion]
			Falls Benutzer mit solcher eMail Adresse nicht existiert oder Passwort eingetippt war, dann bekommt Benutzer eine entsprechende Fehlermeldung. Bei erfolgreicher Anmeldung wird der Benutzer  automatisch zu einer neue Seite umgeleitet wo er Spiele erzeugen, seine Statistiken angucken oder anhand Benutzername einen Spieler suchen kann.
		\end{description}
		
	\end{enumerate}
	
	\item[T1020]  \textbf{\textit{Gastzugang: }} 
	\begin{enumerate}
		\item 
		\begin{description}
			\item[Stand] App ist offen und Benutzer befindet sich auf der Hauptseite des App-s
			\item[Aktion] Benutzer drückt auf \textbf{\textit{Gastspiel}} Knopf, was sich auf untere Mitte der Hauptseite befindet.
			\item[Reaktion] Benutzer befindet sich auf der Seite wo er Spiele erzeugen kann.
		\end{description} 
		
	\end{enumerate}
	\item[T1030] \textbf{\textit{Abmelden: }} 
	\begin{enumerate}
		\item 
		\begin{description}
			\item[Stand] Benutzer ist angemeldet und befindet sich auf der Hauptseite des Apps 
			\item[Aktion] Benutzer drückt auf \textbf{\textit{Abmelden}} Knopf
			\item[Reaktion] Benutzer befindet sich auf der Hauptseite und kann sich wieder anmelden oder als Gastspiel den Spiel betreten.
		\end{description}
	\end{enumerate}
	
	\item[T1040] \textbf{\textit{Anzeige des eigenen, persönlichen Profils: }} 
	\begin{enumerate}
		\item 
		\begin{description}
			\item[Stand] Benutzer hat sich schon angemeldet und befindet sich auf der Seite wo er Spiele erzeugen , Spieler suchen und seine Daten angucken kann
			\item[Aktion] Benutzer drückt auf den Knopf \textbf{\textit{Spielerdaten}}
			\item[Reaktion] Es öffnet sich eine neue Seite wo Benutzer seine Statistiken sieht
		\end{description}
	\end{enumerate}
	
	\item[T1050] \textbf{\textit{Suche nach Benutzern anhand Benutzernamen und Anzeigen der persönlichen Profile anderer Benutzer: }} 
	\begin{enumerate}
		\item 
		\begin{description}
			\item[Stand] Benutzer Befindet sich auf der Seite wo er anhand Benutzernamen andere Benutzer finden kann
			\item[Aktion] Benutzer gibt \textbf{\textit{mustermann }} ein
			\item[Reaktion] Falls \textbf{\textit{mustermann }} noch nicht im System vorhanden ist dann bekommt Benutzer eine entsprechende Meldung, ansonsten öffnet sich eine neue Seite wo Benutzer den anderen Benutzer herausfordern und seine Daten angucken kann.
		\end{description}
	\end{enumerate}
	
	\item[T1060] \textbf{\textit{Multiplayer auf einem Gerät: }} 
	\begin{enumerate}
		\item 
		\begin{description}
			\item[Stand] App ist offen 
			\item[Aktion] Benutzer drückt auf dem Knopf Multiplayer 
			\item[Reaktion] Es öffnet sich ein neues Spielbrett wo zwei Spieler auf einem Gerät gegenseitig Spielen können
		\end{description}
	\end{enumerate}
	
	\item[T2010] \textbf{\textit{Eröffnung eines Spieles: }} 
	\begin{enumerate}
		\item 
		\begin{description}
			\item[Stand] Benutzer befindet sich auf der Seite wo er Spiele erzeugen kann
			\item[Aktion] Benutzer klickt auf den Button \textbf{\textit{SofortSpiel}} 
			\item[Reaktion] Entweder wartet der Benutzer bis auf ein andere Benutzer genauso einen Spiel erzeugt (dann fängt das Spiel zwischen zwei Benutzern an) oder kann der Benutzer den Warteprozess jederzeit abbrechen.
		\end{description}
	\end{enumerate}
	
	\item[T2020] \textbf{\textit{Herausfordern und Spieleinstellungen bestimmen: }} 
	\begin{enumerate}
		\item 
		\begin{description}
			\item[Stand] App ist offen und der Benutzer will einen anderen Benutzer der gerade online ist und sich in keinem Spiel befindet, zum Spiel herausfordern
			\item[Aktion] Benutzer klickt auf den Button \textbf{\textit{Herausfordern }} und wählt gewünschte Farbe und Zeitgrenze
			\item[Reaktion] Herausgeforderte bekommt eine Meldung mit Benutzernamen des Herausforderers und mit einer Zeitgrenze der Herausforderer bestimmt hat. Falls kurz davor Herausgeforderte den App geschlossen hat oder einen Spiel mit jemandem angefangen hat dann bekommt Herausforderer eine entsprechende Meldung
		\end{description}
	\end{enumerate}
	
	\item[T2030] \textbf{\textit{Annehmen einer Herausforderung: }} 
	\begin{enumerate}
		\item 
		\begin{description}
			\item[Stand] Benutzer bekommt eine Meldung dass er von einem anderen Spieler herausgefordert wurde
			\item[Aktion] Benutzer klickt auf den Button \textbf{\textit{Annehmen }} 
			\item[Reaktion] Falls der Herausforderer kurz davor dass Herausgeforderte seine Herausforderung zum Spiel angenommen hat, den App geschlossen hat dann bekommt Herausgeforderte entsprechende Meldung, ansonsten fängt das Spiel an 
		\end{description}
	\end{enumerate}
	
	\item[T2040] \textbf{\textit{Ablehnen einer Herausforderung: }} 
	\begin{enumerate}
		\item 
		\begin{description}
			\item[Stand] Benutzer bekommt eine Meldung dass er von einem anderen Spieler herausgefordert wurde
			\item[Aktion] Benutzer klickt auf den Button \textbf{\textit{Ablehnen }} 
			\item[Reaktion] Falls der Herausforderer kurz davor dass Herausgeforderte seine Herausforderung zum Spiel angenommen hat, die App geschlossen hat, dann bekommt Herausgeforderte entsprechende Meldung, ansonsten bekommt Herausforderer entsprechende Meldung dass Herausgeforderte die Herausforderung zum Spiel abgelehnt hat
		\end{description}
	\end{enumerate}
	
	\item[T3010] \textbf{\textit{Schachfiguren bewegen: }} 
	\begin{enumerate}
		\item 
		\begin{description}
			\item[Stand] Ein Spiel ist am Laufen und Benutzer ist am Zug
			\item[Aktion] Benutzer klickt auf eine  Figur
			\item[Reaktion] Falls die Figur seine eigene Figuren waren, dann es werden alle mögliche Zugfelder markiert, ansonsten wird nichts passieren.
		\end{description}
		\item 
		\begin{description}
			\item[Stand] Eine Schachfigur wurde angeklickt
			\item[Aktion] Der Benutzer klickt auf ein  Feld.
			\item[Reaktion] Falls es ein markiertes Feld war dann wird die Figur gezogen und die Daten an den Server gesendet, ansonsten es passiert nichts, die Figur wird nicht gezogen.
		\end{description}
	\end{enumerate}
	
	
	\item[T3020] \textbf{\textit{Remis bieten: }} 
	\begin{enumerate}
		\item 
		\begin{description}
			\item[Stand] Ein Spiel ist am Laufen und Benutzer ist nicht am Zug
			\item[Aktion] Ein Spieler bietet Remis an
			\item[Reaktion] Der andere Spieler erhält eine Benachrichtigung mit Auswahlmöglichkeit
		\end{description}
	\end{enumerate}
	
	\item[T3030] \textbf{\textit{Remis annehmen: }} 
	\begin{enumerate}
		\item 
		\begin{description}
			\item[Stand] Remis wurde angeboten 
			\item[Aktion] Der Spieler klickt auf den Button \textbf{\textit{Annehmen}} 
			\item[Reaktion] Es erscheint eine Benachrichtigung, die Elozahlen und Statistiken der Spieler werden aktualisiert
		\end{description}
	\end{enumerate}
	
	\item[T3040] \textbf{\textit{Remis ablehnen: }} 
	\begin{enumerate}
		\item 
		\begin{description}
			\item[Stand] Ein Spiel ist am Laufen
			\item[Aktion] Der Spieler klickt auf den Button \textbf{\textit{Aufgeben}} 
			\item[Reaktion] Es erscheint eine Benachrichtigung, die Elozahlen und Statistiken der Spieler werden aktualisiert
		\end{description}
	\end{enumerate}
	
	\item[T3050] \textbf{\textit{Rückkampf anbieten: }} 
	\begin{enumerate}
		\item 
		\begin{description}
			\item[Stand] Spiel ist beendet 
			\item[Aktion] Eine von beiden Spieler drücken auf dem Button \textbf{\textit{Rückkampf}} 
			\item[Reaktion] Anderer Spieler bekommt eine Benachrichtigung über den Rückkampf Angebot
		\end{description}
	\end{enumerate}
	
	
\end{description}



\pagebreak
\section{Systemmodelle}
\subsection{Szenarien}
\textbf{Szenario 1: \glqq Spiele gegen zufälligen Spieler\grqq} \\
Max Mustermann möchte eine Runde \Gls{Schach} in seiner Mittagspause spielen. Er holt sein \gls{Smartphone} aus der Tasche und klickt auf das Appsymbol.
Da er die App noch nie verwendet hat, muss er sich zuerst anmelden. Nachdem er dies getan hat, befindet er sich im Hauptmenü.
Da keiner seiner Kollegen Zeit hat, möchte er gegen einen zufälligen Gegner spielen. Nun klickt er auf \glqq Spiel suchen\grqq und bekommt die Meldung, dass er nun der Spielsuche hinzugefügt wurde.
Nach kurzer Zeit bekommt er einen Gegner zugewiesen und die Partie beginnt. Vor ihm erscheint das \gls{Schachbrett} in seiner \gls{Ausgangsposition}.

Herr Mustermann hat die Farbe Weiß zugewiesen bekommen und sein Gegner Schwarz. Er wählt einen Bauern an und bekommt seine möglichen Züge dieser \gls{Schachfigur} angezeigt.
Max Mustermann macht seinen Zug. Als nächstes zieht sein Gegner. Nun ist er wieder am Zug. Beide ziehen nun immer abwechselnd. Nach 35 Zügen hat Herr Mustermann
seinen Gegner \gls{Schachmatt} gesetzt. Die Partie ist somit beendet und Max Mustermann bekommt die Meldung, dass er gewonnen hat. Ihm steht nun zur Auswahl ob er eine Revanche anbieten möchte, die Partie auf seinem Gerät speichern will, oder einfach nur ins Hauptmenü möchte. Er klickt auf den Button \glqq Zum Hauptmenü\grqq und wird zum Hauptmenü weitergeleitet.

\textbf{Szenario 2: \glqq Spiele gegen einen Freund\grqq} \\
Magnus und Fabiano wollen in ihrer Freizeit gegeneinander Schach spielen, haben aber gerade kein Schachbrett zur Hand. Beide öffnen auf ihren \gls{Android}-Gerät ihre Schach-App, und Magnus klickt auf den Button \glqq Spieler suchen\grqq. Es öffnet sich eine Spielerübersicht mit einer Suchleiste, in welcher er den Spielernamen von Fabiano eingibt. In der Spielerübersicht erscheint Fabianos Profil, welches Magnus anklickt. Daraufhin öffnet sich Fabianos Profilübersicht mit seinen Statistiken. Auf Fabianos Profil betätigt Magnus den Button \glqq Herausfordern\grqq, und es erscheint ein Auswahlfenster. Magnus wählt als \gls{Bedenkzeit} 2 Stunden und die zufällige Farbenverteilung. Fabiano erhält daraufhin eine Mitteilung, dass er von Magnus zu einem Spiel mit \gls{Bedenkzeit} 2 Stunden herausgefordert wurde, und hat die Möglichkeit, dieses anzunehmen oder abzulehnen. Er klickt auf \glqq Annehmen\grqq und beiden Spielern öffnet sich ein \gls{Schachbrett} in \gls{Ausgangsposition}.

Magnus bekommt die schwarzen Figuren zugeteilt, Fabiano die weißen. Nach 115 Zügen ist die Stellung immer noch ausgeglichen und Fabiano klickt auf \glqq Remis anbieten\grqq. Magnus erhält eine Benachrichtigung mit der Möglichkeit, das \gls{Remis} anzunehmen oder abzulehnen. Auch er sieht in der Stellung keine Gewinnmöglichkeit und klickt deshalb auf \glqq Akzeptieren\grqq. Das Spiel wird beendet und die Statistiken beider Spieler werden aktualisiert. Beide Spieler erhalten eine Mitteilung mit der Möglichkeit, eine \gls{Revanche} zu fordern und das Spiel abzuspeichern. Da Magnus das Spiel gerne noch einmal ansehen möchte, klickt er auf \glqq Spiel speichern\grqq und erhält eine Textdatei mit der algebraischen \gls{Notation} der Partie auf sein \gls{Smartphone}.

\textbf{Szenario 3: \glqq Statistiken einsehen\grqq} \\
Magnus behauptet, er wäre ein besserer Spieler als Fabiano. Dieser zweifelt das an, und bittet Magnus, dessen Statistiken mit seinen zu vergleichen. Beide öffnen ihre Schachapp und sind direkt mit ihrem Account angemeldet, da sie dies vorher eingestellt haben. Anschließend klicken sie im Hauptmenü ihrer Schach-App auf \glqq Statistiken\grqq. Sie können nun jeweils ihre Anzahl an gespielten, gewonnenen, remisierten und verlorenen Spielen, sowie ihre \gls{Elo}zahl einsehen. Beide kommen zu dem Schluss, dass Magnus bessere Werte vorzuweisen hat.
\pagebreak


\subsection{Anwendungsfalldiagramme} 
		\begin{minipage}{\linewidth}
			\centering
			\includegraphics[width=1\linewidth]{Anwendung}
			\label{fig:Schach-App}
				\captionof{figure}{Schach-App}
		\end{minipage}

\pagebreak
\glsaddall 
\printglossaries

\end{document}
