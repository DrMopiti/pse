\documentclass[parskip=full]{scrartcl}
\usepackage[utf8]{inputenc} % use utf8 file encoding for TeX sources 

\usepackage[T1]{fontenc} % avoid garbled Unicode text in pdf 
\usepackage[german]{babel} % german hyphenation, quotes, etc 
\usepackage{hyperref} % detailed hyperlink/pdf configuration
\usepackage{graphicx}
\usepackage[toc]{glossaries}
\usepackage{caption}
\usepackage{pdfpages}
\hypersetup{ % ‘texdoc hyperref‘ for options 
	pdftitle={Entwurf}, %
	bookmarks=true,%
}
\usepackage{csquotes} % provides \enquote{} macro for "quotes"
\usepackage{enumitem}

\begin{document}
	\begin{titlepage}
		
		\centering
		\vspace*{0.2\textheight}
		{\Large Praxis der Softwareentwicklung}\\[\baselineskip]
		\vspace{2cm}
		{\Huge \textbf{Qualitätssicherung}}\\[\baselineskip]\par
		\vspace{2cm}
		{\LARGE Daniel Helmig, Tim Groß, Rukiye Devran, Orkhan Aliev}\par		
		\newpage	
		\tableofcontents
		\pagebreak
		
	\end{titlepage}
	\section{Kriterien}
\subsection{Musskriterien}
\begin{description}
	\item[KM1010] \textbf{Schachregeln}
	\begin{itemize}
		\item Schachregeln sind implementiert mit Ausnahme eines Unentschiedens aufgrund von Materialmangel.
		\item Das Spielende wird erkannt. Die Korrektheit wurde mit JUnit Testfällen mit hoher Abdeckung überprüft.
	\end{itemize}
	\item[KM1020] \textbf{\textit{Einhaltung der Schachregeln}}
	\begin{itemize}
		\item Der Nutzer bekommt nur gültige Züge angezeigt und kann nur diese anwenden.
		\item Dies wurde mit JUnit Testfällen und mehreren Testspielen verschiedener Testpersonen überprüft.
	\end{itemize}
	\item[KM1030] \textbf{\textit{Spielsuche}}
	\begin{itemize}
		\item Die Spielsuche wurde im Vergleich zum Pflichtenheft abgeändert.
		Der Button Sofortspiel startet ein Offline-Spiel auf einem Gerät. Die schwarzen Figuren werden hierbei auf den Kopf gedreht.
		Durch klicken des Buttons Spielersuche, werden alle Spieler aufgelistet.
		Wählt man einen Spieler an, sieht man dessen Online-Status und kann ihn, falls dieser online ist, herausfordern. Der Spieler der auf Herausfordern klickt, bekommt die weißen Figuren zugewiesen.
	\end{itemize}
	\item[KM1040] \textbf{GUI}
	\begin{itemize}
		\item Die Oberfläche besteht aus:
		\begin{itemize}		
			\item Login-Seite: Der Nutzer vergibt sich hier selbstständig einen Nutzernamen.
			
			\item Hauptmenü: Hier kann der Nutzer ein Offline-Spiel starten, einen anderen Spieler herausfordern oder seine Statistiken einsehen.
			
			\item Statistik-Seite: Der Nutzer kann seine Spielstatistiken einsehen.
			
			\item Schachbrett: Hier kann der Nutzer in einer Partie seine Züge durchführen.
			
		\end{itemize}
	\end{itemize}
	\item[KM1050] \textbf{\textit{Schachfigur bewegen}}
	\begin{itemize}
		\item Figuren können durch antippen ausgewählt werden.
		\item Wird eine Figur angewählt, werden ihre gültigen Züge angezeigt.
		\item Es können nur gültige Züge ausgeführt werden.
		\item Die Zugberechnung wurde durch JUnit Tests überprüft, die korrekte Darstellung in mehreren Testspielen überprüft.
	\end{itemize}
\end{description}
\subsection{Wunschkriterien}
\begin{itemize}
\item{\textbf{Account/Gastzugang:}} Für das Benutzen der App wird ein Name benötigt, welcher beim ersten Öffnen eingegeben wird. 
\item{\textbf{Zwei Spieler auf einem Gerät:}} Wurde implementiert. Hierfür werden die schwarzen Figuren um 180 Grad gedreht dargestellt. \\
\\
Ansonsten wurden keine Wunschkriterien implementiert.
\end{itemize}
	\section{Funktionale Anforderungen}
\subsection{Benutzerfunktionen}
\begin{description}
	\item[F1010] \textbf{\textit{Anmelden: }} Ein Android Nutzer der auch einen Google Account besitzt, kann sich auf der Hauptseite der App erfolgreich anmelden. Für die Anmeldung im System sind folgende Angaben erforderlich:
	\begin{itemize}
		\item e-Mail Adresse, die mit dem Google Konto in Verbindung steht
		\item Passwort des Google Kontos
	\end{itemize}  
	\item \textbf{Ergebnis: } Abgeändert. Die Anmeldung erfolgt über einen vom Nutzer selbst gewählten Namen.

	\item[F1020] \textbf{\textit{Gastzugang: }} Benutzer die keinen Google Account besitzen, können sich als Gast im System anmelden. Bei der Anmeldung wird ihnen ein eindeutiger Benutzername vom System zugewiesen.
	\item \textbf{Ergebnis: } Entfällt, wie in Testfall T1010 beschrieben.
	
	\item[F1030] \textbf{\textit{Abmelden: }} Benutzer, die sich bereits mit ihrem Google Account angemeldet haben, können sich wieder vom System abmelden.
	\item \textbf{Ergebnis: } Entfällt, wie in Testfall T1010 beschrieben.
	
	
	\item [F1040] \textbf{\textit{Rangliste und eigene Statistiken anzeigen lassen: }} Mit einem Klick auf \textbf{\textit{Rangliste/Statistik }} ist der Benutzer in der Lage seine eigenen Statistiken aufzurufen. Als Statistiken sieht der Benutzer folgende Angaben:
	
	\begin{itemize}
		\item Anzahl der Spiele
		\item Anzahl der gewonnenen Spiele
		\item Anzahl der verlorenen Spiele
		\item Gewinnrate
		\item Anzahl der unentschiedenen Spiele
		\item Elo-Zahl		
	\end{itemize}
	\item \textbf{Ergebnis: } Bis auf die Elo-Zahl werden alle oben angegebenen Daten angezeigt.
	
	\item[F1050] \textbf{\textit{Suche nach Benutzern: }} Der Benutzer kann mit der Suchfunktion nach anderen Benutzern des Systems anhand ihres Benutzernamens suchen.
	\item \textbf{Ergebnis: } Es werden alle Benutzer seit Server-Start aufgelistet.
	
	\item[F1060]  \textbf{\textit{Anzeigen der persönlichen Profile anderer Benutzer: }}
	Der Benutzer kann sich die persönlichen Profile von anderen Benutzern anzeigen lassen, wobei er den Benutzernamen und die Statistiken sieht. Die anderen Benutzer können sich genauso sein Profil anzeigen lassen.
	\item \textbf{Ergebnis: } Es ist nicht möglich sich Statistiken von anderen Spielern anzeigen zu lassen.
	
	\item[F1070] \textbf{\textit{Chatten: }} Spieler können während des Spieles miteinander chatten.	
	\item \textbf{Ergebnis: } Nicht implementiert.
	
\end{description}

\subsection{Initialisierung}
\begin{description}
	\hypertarget{F2010}{\item[F2010]}\textbf{\textit{Eröffnung eines Spieles: }} Der Benutzer kann Spiele erzeugen, ohne dabei einen anderen Benutzer als Gegner angeben zu müssen. Dann bekommt er vom System einen Gegner zugewiesen, der ebenfalls ein Spiel erzeugt hat. Die Farbe ist zufällig und es gibt eine feste Bedenkzeit von 15 Minuten für jeden Spieler.
	\item \textbf{Ergebnis: } Der Nutzer kann offline spielen, oder einen anderen Spieler für ein Online-Spiel auswählen. Der Herausforderer bekommt weiß zugewiesen und es gibt keine Bedenkzeit.
	
	\item[F2020] \textbf{\textit{Herausfordern: }} Nachdem ein entsprechender Gegner ausgesucht wurde \textbf{\textit{F1050}} kann der Benutzer ihn zum Spiel herausfordern.
	\item \textbf{Ergebnis: } Dies ist soweit möglich.
	
	\item [F2030] \textbf{\textit{Annehmen einer Herausforderung: }} Der Benutzer kann die Herausforderung zum Spiel \textbf{\textit{F2030}} annehmen.
	\item \textbf{Ergebnis: } Die Herausforderung wird implizit angenommen.
	\item [F2040] \textbf{\textit{Ablehnung einer Herausforderung: }} Der Benutzer kann die Herausforderung zum Spiel \textbf{\textit{F2030}} ablehnen.
	\item \textbf{Ergebnis: } Die Herausforderung wird implizit angenommen.
	
	\item[F2050] \textbf{\textit{Spieleinstellungen bestimmen: }} Bei der Herausforderung einer bestimmten Person zum Spiel \textbf{\textit{F2020}} kann der Herausforderer folgende Spieleinstellungen aufstellen:
	\begin{enumerate}
		\item Bedenkzeit
		\item Farbe
	\end{enumerate}
	\item \textbf{Ergebnis: } Es können keine weiteren Einstellungen vorgenommen werden.
	
	\item[F2060] \textbf{\textit{Multiplayer auf einem Gerät: }}  Mit einer Multiplayer Funktion können zwei Spieler auf einem Gerät gegeneinander spielen.
	\item \textbf{Ergebnis: } Die Möglichkeit zu zweit auf einem Gerät zu spielen wurde implementiert.
	\item \textbf{Test-Art: } Durch mehrere Testpersonen und Testspiele überprüft.
	
	
\end{description}


\subsection{Spielverlauf} 
\begin{description}
	\item[F3010]\textbf{\textit{Schachfiguren bewegen: }}Der Benutzer kann während des Spieles, falls er an der Reihe ist, einen Zug seiner Wahl unter Beibehaltung der Schachregeln ausführen, wobei dem Benutzer alle erlaubte Züge  angezeigt werden.
	\item \textbf{Ergebnis: } Der Testfall T3010 funktioniert wie vorgesehen. Der Benutzer kann, wenn er an der Reihe ist, eine Figur auswählen und die möglichen Zielfelder werden markiert. Dann kann der Benutzer eines dieser Felder auswählen und der Zug wird ausgeführt.
	
	\item[F3020] \textbf{\textit{Unentschieden(Remis) bieten: }} Ein Spieler der Remis anbieten möchte, tut dies, nachdem er einen Zug auf dem Schachbrett ausgeführt und nachdem seine Uhr angehalten und sich die des Gegners in Gang gesetzt hat.
	\item \textbf{Ergebnis: } Es ist nicht möglich ein Remis anzubieten.
	
	\item[F3030] \textbf{\textit{Remis annehmen: }} Der Benutzer kann, wenn ihm Unentschieden angeboten wurde \textbf{\textit{F3020}}, das Angebot annehmen.
	\item \textbf{Ergebnis: } Es ist nicht möglich ein Remis anzunehmen.
	
	\item[F3040] \textbf{\textit{Remis ablehnen: }} Der Benutzer kann, wenn ihm Unentschieden angeboten wurde \textbf{\textit{F3020}}, das Angebot ablehnen.
	\item \textbf{Ergebnis: } Es ist nicht möglich ein Remis abzulehnen.	
	
	\item[F3050] \textbf{\textit{Aufgeben: }} Während des Spiels kann jeder Spieler jederzeit aufgeben.
	\item \textbf{Ergebnis: } Es ist nicht möglich aufzugeben.
	
	\item[F3060] \textbf{\textit{Spiel beenden: }} Ein Spiel endet, falls einer von beiden Spielern den anderen Schachmatt gesetzt hat, ein Spieler aufgegeben hat, oder eine von unten gezählten Remis Situation auftritt:
	\begin{enumerate}	    
		\item Falls einer von beiden Spielern Remis angeboten und der andere Spieler das Angebot angenommen hat
		\item Falls ein Patt auftritt
		\item Falls eine tote Stellung vorliegt
		\item Wenn eine identische Stellung mit gleichen Zugmöglichkeiten und demselben Spieler am Zug mindestens zum dritten Mal auf dem Schachbrett entstanden ist 
	\end{enumerate}
	\item \textbf{Ergebnis: } Bei einem Patt oder nach 50 Zügen ohne Bauernbewegung/Figuren schlagen wird das Spiel mit unentschieden beendet.
	
	\item[F3070] \textbf{\textit{Rückkampf anbieten: }} Nachdem ein Spiel zu Ende gekommen ist, muss jeder Spieler den anderen Spieler einen Rückkampf anbieten können.	
	\item \textbf{Ergebnis: } Es ist nicht möglich einen Rückkampf anzubieten.
		
	\item[F3080] \textbf{\textit{Spiel speichern: }}  Jeder Spieler kann schon beendete Spiele nach Spielende sofort auf dem Gerät speichern. 
	\item \textbf{Ergebnis: } Es ist nicht möglich eine beendete Partie zu speichern.
	
\end{description}
	\section{Nichtfunktionale Anforderungen}
\begin{description}
	
	\item[NF1010] \textbf{\textit{Start der App: }} Das Starten der App soll auf aktuellen Geräten maximal 3 Sekunden benötigen. 
	\item \textbf{Ergebnis: } Die App braucht weniger als 3 Sekunden zum Start.
	\item \textbf{Test-Art: } Zehnmaliger App Start mit einer variierender Anzahl an bereits geöffneten Apps. Manuell mit Stoppuhr gemessen.
	
	\item[NF1020] \textbf{\textit{Spiel Erstellung: }} Der Erstellungsprozess \hyperlink{F2010}{\textbf{\textit{F2010}}} einer neuen Partie darf nicht länger als 5 Sekunden dauern.
	\item \textbf{Ergebnis: } Die Erstellung braucht zwischen zwei und sechs Sekunden, je nach Internetverbindung.
	\item \textbf{Test-Art: } Sechs manuelle Spielerstellungen, drei mit aktiver WLAN-Verbindung, drei mit mobiler Datenverbindung. Mit Stoppuhr gemessen.
	
	\item[NF1030] \textbf{\textit{Ermittlung von Spielzügen: }} Die Ermittlung \hyperlink{F3010}{\textbf{\textit{F3010}}} an möglichen gültigen Zügen darf nicht länger als 0,1 Sekunden dauern.
	\item \textbf{Ergebnis: } Aufgrund der kleinen Zeitspanne schwer zu testen, da die Reaktionsgeschwindigkeit Messfehler verursacht.
	\item \textbf{Test-Art: } Zu verschiedenen Zeitpunkten eines Offline-Spieles getestet und Zeit per Hand gemessen. Eventuell optische Erkennung des Displays besser geeignet.
	
	\item[NF1040] \textbf{\textit{Herausforderung: }} Falls ein Spieler herausgefordert wird, hat dieser zwei Minuten Zeit diese anzunehmen. Danach verfällt die Herausforderung. 
	\item \textbf{Ergebnis: } Wird ein Spieler herausgefordert, wird die Herausforderung implizit akzeptiert.
	\item \textbf{Test-Art: } Entfällt.
	
	\item[NF1050] \textbf{\textit{Weiterleitung von Nachrichten: }} Die Weiterleitung und Überprüfung einzelner Züge soll auf dem Server nicht länger als 2 Sekunden dauern.
	\item \textbf{Ergebnis: } Längere Wartezeiten konnten bisher nicht festgestellt werden.
	\item \textbf{Test-Art: } Benutzen der App zu Testzwecken.
	
	\item[NF1060] \textbf{\textit{Verlust von Paketen: }} Bei Übertragungen zwischen zwei Geräten sollen keine Pakete unbemerkt verloren gehen.
	\item \textbf{Ergebnis: } Bisher keine Paketverluste bemerkbar.
	\item \textbf{Test-Art: } Benutzen der App zu Testzwecken.
	
	\item[NF1070] \textbf{\textit{Manipulation von Nachrichten: }} Nachrichten/Spielzüge sollen unverändert am Empfänger eintreffen. Sollten Änderungen vorgenommen worden sein, soll dies vom Empfänger erkannt werden können.
	\item \textbf{Ergebnis: } Es wird beim Empfänger nicht auf Korrektheit überprüft.
	\item \textbf{Test-Art: } Entfällt.
	
	\item[NF1080] \textbf{\textit{Wartung: }} Der Server soll wartungsfrei und ohne Neustarts auskommen.
	\item \textbf{Ergebnis: } In dieser vergleichsweise kurzen Zeitspanne wurde der Server nicht neu gestartet. Dieses Ergebnis liefert jedoch keine zuverlässige Aussage.
	\item \textbf{Test-Art: } Entfällt. Theoretisch mit regelmäßigen Ping Anfragen überprüfbar.
	
	
\end{description}

\section{Globale Testfälle}
\subsection{Benutzerfunktionen}
\begin{description}
	\item[T1010] \textbf{\textit{Anmelden: }} 
	\begin{description}
	\item Wurde abgewandelt in eine lokale Anmeldung und der Button Anmelden wurde nicht implementiert, da kein Gastzugang existiert.
	Beim erstmaligen Start gibt der Nutzer einen selbst gewählten Namen ein.
	\end{description}	

	\item[T1020]  \textbf{\textit{Gastzugang: }} 
    \begin{description}
	\item Nicht implementiert.
	\end{description}

	\item[T1030] \textbf{\textit{Abmelden: }}
	\begin{description}
	\item Nicht implementiert, damit der Account Gerät-gebunden ist.
	\end{description}
	
	\item[T1040] \textbf{\textit{Anzeige des eigenen, persönlichen Profils: }} 
	\begin{description}
	\item Funktioniert wie vorgesehen. 
	\end{description}
	
	\item[T1050] \textbf{\textit{Suche nach Benutzern anhand des Benutzernamens und Anzeigen der persönlichen Profile anderer Benutzer: }} 
	\begin{description}
	\item Falls der Spieler nicht vorhanden ist wird keine Meldung angezeigt, dass er nicht vorhanden ist.  Er wird gar nicht in der Liste angezeigt und man muss einen anderen vorhandenen Spieler suchen. \\
	\item Es werden alle Spieler seit Server Start aufgelistet, der Spieler selbst ausgenommen.
	\end{description}
	
	
	\item[T1060] \textbf{\textit{Multiplayer auf einem Gerät: }} 
	\begin{description}
	\item Der Button Multiplayer wurde in Sofortspiel umbenannt und die Figuren sind dementsprechend für die Spieler umgedreht.
	Durch mehrere Testpersonen getestet.
	\end{description}
	
	
	\subsection{Initialisierung}
	\item[T2010] \textbf{\textit{Eröffnung eines Spieles: }} 
	\begin{description}
	\item Der Button Sofortspiel erzeugt ein Offline-Spiel, bei dem zwei Spieler auf einem Gerät gegeneinander spielen. Durch die Spieler-Suche bekommt man eine Liste aller Spieler und kann diese herausfordern, wenn weder der Spieler selber, noch der Gegner bereits in einem Spiel sind.
	\end{description}
	

	
	\item[T2020] \textbf{\textit{Herausfordern und Spieleinstellungen bestimmen: }} 
	\begin{description}
	\item Gewünschte Farbe und Bedenkzeit wurden nicht implementiert.
	\item Der Nutzer der einen anderen Nutzer herausfordert, bekommt weiß zugewiesen.
	\end{description}
	

	
	\item[T2030] \textbf{\textit{Annehmen einer Herausforderung: }} 
	\begin{description}
	\item Nicht implementiert, ein Spiel wird automatisch angenommen.
	\end{description}

	
	\item[T2040] \textbf{\textit{Ablehnen einer Herausforderung: }} 
	\begin{description}
	\item Nicht implementiert.
	\end{description}

	
	\subsection{Spielverlauf}
	
	\item[T3010] \textbf{\textit{Schachfiguren bewegen: }} 
	\begin{description}
	\item Funktioniert wie vorgesehen. Wurde mit mehreren Testpersonen getestet.
	\end{description}
			
	\item[T3020] \textbf{\textit{Remis bieten: }} 
	\begin{description}
	\item Nicht implementiert.
	\end{description}

	\item[T3030] \textbf{\textit{Remis annehmen: }}
	\begin{description}
	\item Nicht implementiert. Siehe [T3020].
	\end{description}
	
	\item[T3040] \textbf{\textit{Remis ablehnen: }}
	\begin{description}
	\item Nicht implementiert.
	\end{description}
	
	\item[T3050] \textbf{\textit{Aufgeben: }} 
	\begin{description}
	\item Nicht implementiert.
	\end{description}
	
	\item[T3060] \textbf{\textit{Rückkampf anbieten: }}
	\begin{description}
		\item Nicht implementiert.
	\end{description}
	
	\item[T3070] \textbf{\textit{Spiel speichern: }} 
	\begin{description}
		\item Nicht implementiert.
	\end{description}
	
	
\end{description}

\section{Bugs}
\begin{itemize}
\item{\textbf{Online-Spielen:}}
	\begin{itemize}
	\item{\textbf{Symptom:}} Online-Spielen war nicht möglich.
	\item{\textbf{Ursache:}} Serverkommunikation im gleichen Thread wie die restliche App laufen zu 	lassen führte zu Fehlern. (NetworkOnMainThreadException)
	\item{\textbf{Behebung:}} Serverkommunikation mittels AsyncTask in separate Threads auslagern. 
	\end{itemize}

\item{\textbf{ Spielersuche:}} 
	\begin{itemize} 
	\item{\textbf{Symptom:}} Spielersuche nicht möglich.
	\item{\textbf{Ursache:}} Netzwerkzugriff im UI-Thread verboten. (NetworkOnMainThreadException)
	\item{\textbf{Behebung:}} Asynchrone Methode enqueue() mit Callback, statt synchroner Methode execute() verwendet. 
	\end{itemize}

\item{\textbf{Verlieren von Weiß:}}
	\begin{itemize}
	\item{\textbf{Symptom:}} Weiß konnte nicht verlieren.
	\item{\textbf{Ursache:}} Es wurde nicht abgefragt, ob nach einem Zug von schwarz das Spiel beendet war.
	\item{\textbf{Behebung:}} Die Ergebnisabfrage wurde in eine extra Methode ausgelagert, die dann an den richtigen Stellen aufgerufen wird.
	\end{itemize}

\item{\textbf{Fehlerhafte Brettübertragung:}}
	\begin{itemize}
		\item{\textbf{Symptom:}} Der Brettstring wurde korrekt in der Datenbank gespeichert, kam aber abgeschnitten in der App an.
		\item{\textbf{Ursache:}} Unbekannt.
		\item{\textbf{Behebung:}} Die Trennung der einzelnen Bereiche im Brettstring mit dem Symbol \glqq X\grqq, statt \glqq \#\grqq \space vorgenommen.
	\end{itemize}

\item{\textbf{Websocket Verbindung ohne Name:}}
\begin{itemize}
	\item{\textbf{Symptom:}} Der Nutzer meldet sich als NoUser beim Server.
	\item{\textbf{Ursache:}} Die Verbindung wurde schon aufgebaut, bevor der Nutzer einen Namen eingegeben hat.
	\item{\textbf{Behebung:}} Es wird überprüft ob der Nutzer schon einen Namen eingegeben hat.
\end{itemize}

\item{\textbf{Doppelte Namen:}}
\begin{itemize}
	\item{\textbf{Symptom:}} Kein Symptom.
	\item{\textbf{Ursache:}} Es wurde nicht geprüft ob der eingegebene Name schon vergeben ist.
	\item{\textbf{Behebung:}} Es wird überprüft ob der eingegebene Name schon vergeben ist und falls ja wird der Login mit diesem Namen unterbunden.
\end{itemize}

\item{\textbf{Herausfordern obwohl man bereits spielt:}}
\begin{itemize}
	\item{\textbf{Symptom:}} Der Nutzer bekam keine Meldung, falls er auf "Herausfordern" geklickt hat, aber bereits in einem Spiel war.
	\item{\textbf{Ursache:}} Es wurde nicht geprüft ob man selber bereits in einem Spiel ist. Der Server hat diese ungültige Abfrage aber abgefangen und kein neues Spiel erstellt. 
	\item{\textbf{Behebung:}} Der Nutzer wird nun benachrichtigt, wenn er sich bereits in einem Spiel befindet.
\end{itemize}


\item{\textbf{Kein Ziehen beim Online-Spiel möglich:}}
\begin{itemize}
	\item{\textbf{Symptom:}} Bei Online-Spielen konnten die Figuren weder bewegt, noch gewählt werden.
	\item{\textbf{Ursache:}} Fehlerhafte Abfragen bei Online-Spielen, die Spielerfarbe wurde nicht korrekt abgefragt, da diese nicht vorhanden war.
	\item{\textbf{Behebung:}} Übergeben der Spielerfarbe beim Start eines Online-Spieles, entfernen falscher Abfragen.
\end{itemize}

\item{\textbf{Remis anbieten und Aufgeben funktionierten gar nicht:}}
\begin{itemize}
	\item{\textbf{Symptom:}} Bei Online-Spielen konnte weder Remis angeboten, noch aufgegeben werden.
	\item{\textbf{Ursache:}} Keine Speicherung von Remisangeboten/Aufgaben im BoardState, sowie fehlende Benachrichtigungsmöglichkeit.
	\item{\textbf{Behebung:}} Speicherung der genannten Anfragen in BoardState, abfragen beim Gegner zur Ausführung entsprechender Aktionen.
\end{itemize}

\item{\textbf{Weiß kann bei Online-Spielen nicht ziehen:}}
\begin{itemize}
	\item{\textbf{Symptom:}} Bei Online-Spielen konnte der weiße Spieler nach dem ersten Zug nicht mehr ziehen, wenn das Brett neu geladen wird.
	\item{\textbf{Ursache:}} Die Farbe beim Laden eines offenen Spiels wurde fest auf schwarz gesetzt.
	\item{\textbf{Behebung:}} Hinzufügen einer Server-Abfrage, welche die eigene Farbe im aktuellen Spiel liefert; Übergeben des Parameters bei Erzeugen eines Brettes.
\end{itemize}

\end{itemize}
\end{document}