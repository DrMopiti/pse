\documentclass[parskip=full]{scrartcl}
\usepackage[utf8]{inputenc} % use utf8 file encoding for TeX sources 

\usepackage[T1]{fontenc} % avoid garbled Unicode text in pdf 
\usepackage[german]{babel} % german hyphenation, quotes, etc 
\usepackage{hyperref} % detailed hyperlink/pdf configuration
\usepackage{graphicx}
\usepackage[toc]{glossaries}
\usepackage{caption}
\usepackage{pdfpages}
\hypersetup{ % ‘texdoc hyperref‘ for options 
	pdftitle={Entwurf}, %
	bookmarks=true,%
}
\usepackage{csquotes} % provides \enquote{} macro for "quotes"
\usepackage{enumitem}

\begin{document}
	\begin{titlepage}
		
		\centering
		\vspace*{0.2\textheight}
		{\Large Praxis der Softwareentwicklung}\\[\baselineskip]
		\vspace{2cm}
		{\Huge \textbf{Qualitätssicherung}}\\[\baselineskip]\par
		\vspace{2cm}
		{\LARGE Rukiye Devran, Tim Groß, Daniel Helmig, Orkhan Aliev}\par		
		\newpage	
		\tableofcontents
		\pagebreak
		
	\end{titlepage}
\section{Musskriterien}
\begin{description}
	\item[KM1010] \textbf{Schachregeln}
	\begin{itemize}
		\item Schachregeln sind implementiert mit Ausnahme eines Patts Aufgrund von Materialmangels.
		\item Das Spielende wird erkannt.
	\end{itemize}
	\item[KM1020] \textbf{\textit{Einhaltung der Schachregeln}}
	\begin{itemize}
		\item Der Nutzer bekommt nur gültige Züge angezeigt und kann nur diese anwenden.
	\end{itemize}
	\item[KM1030] \textbf{\textit{Spielsuche}}
	\begin{itemize}
		\item Die Spielsuche wurde im Vergleich zum Pflichtenheft abgeändert.
		Der Button Sofortspiel startet ein Offline-Spiel auf einem Gerät. Die schwarzen Figuren werden hierbei auf den Kopf gedreht.
		Durch klicken des Buttons Spielersuche, werden alle Spieler aufgelistet.
		Wählt man einen Spieler an, sieht man dessen Online-Status und kann ihn, falls dieser Online ist, herausfordern. Der Spieler der auf "Herausfordern" klickt, bekommt die weißen Figuren zugewiesen.
	\end{itemize}
	\item[KM1040] \textbf{GUI}
	\begin{itemize}
		\item Die Oberfläche besteht aus:
		\begin{itemize}		
			\item Login Seite: Der Nutzer vergibt sich hier selbstständig einen Nutzernamen.
			
			\item Hauptmenü: Hier kann der Nutzer ein Offline-Spiel starten, einen anderen Spieler heruasfordern oder seine Statistiken einsehen.
			
			\item Statistik Seite: Der Nutzer kann seine Spielstatistiken einsehen.
			
			\item Schachbrett: Hier kann der Nutzer in einer Partie seine Züge durchführen.
			
		\end{itemize}
	\end{itemize}
	\item[KM1050] \textbf{\textit{Schachfigur bewegen}}
	\begin{itemize}
		\item Figuren können durch antippen ausgewählt werden.
		\item Wird eine Figur angewählt, werden ihre gültigen Züge angezeigt.
		\item Es können nur gültige Züge ausgeführt werden.
	\end{itemize}
\end{description}
\section{Wunschkriterien}
\section{Globale Testfälle}
\end{document}