\documentclass[parskip=full]{scrartcl}
\usepackage[utf8]{inputenc} % use utf8 file encoding for TeX sources 

\usepackage[T1]{fontenc} % avoid garbled Unicode text in pdf 
\usepackage[german]{babel} % german hyphenation, quotes, etc 
\usepackage{hyperref} % detailed hyperlink/pdf configuration
\usepackage{graphicx}
\usepackage[toc]{glossaries}
\usepackage{caption}
\usepackage{pdfpages}
\hypersetup{ % ‘texdoc hyperref‘ for options 
	pdftitle={Entwurf}, %
	bookmarks=true,%
}
\usepackage{csquotes} % provides \enquote{} macro for "quotes"
\usepackage{enumitem}

\begin{document}
	\begin{titlepage}
		
		\centering
		\vspace*{0.2\textheight}
		{\Large Praxis der Softwareentwicklung}\\[\baselineskip]
		\vspace{2cm}
		{\Huge \textbf{Qualitätssicherung}}\\[\baselineskip]\par
		\vspace{2cm}
		{\LARGE Tim Groß, Daniel Helmig}\par		
		\newpage	
		\tableofcontents
		\pagebreak
		
	\end{titlepage}
	\section{Kriterien}
\subsection{Musskriterien}
\begin{description}
	\item[KM1010] \textbf{Schachregeln}
	\begin{itemize}
		\item Schachregeln sind implementiert mit Ausnahme eines Patts Aufgrund von Materialmangels.
		\item Das Spielende wird erkannt.
	\end{itemize}
	\item[KM1020] \textbf{\textit{Einhaltung der Schachregeln}}
	\begin{itemize}
		\item Der Nutzer bekommt nur gültige Züge angezeigt und kann nur diese anwenden.
		\item Dies wurde mit JUnit Testfällen für die Schachlogik und mehreren Testspielen verschiedener Testpersonen überprüft.
	\end{itemize}
	\item[KM1030] \textbf{\textit{Spielsuche}}
	\begin{itemize}
		\item Die Spielsuche wurde im Vergleich zum Pflichtenheft abgeändert.
		Der Button Sofortspiel startet ein Offline-Spiel auf einem Gerät. Die schwarzen Figuren werden hierbei auf den Kopf gedreht.
		Durch klicken des Buttons Spielersuche, werden alle Spieler aufgelistet.
		Wählt man einen Spieler an, sieht man dessen Online-Status und kann ihn, falls dieser Online ist, herausfordern. Der Spieler der auf "Herausfordern" klickt, bekommt die weißen Figuren zugewiesen.
	\end{itemize}
	\item[KM1040] \textbf{GUI}
	\begin{itemize}
		\item Die Oberfläche besteht aus:
		\begin{itemize}		
			\item Login Seite: Der Nutzer vergibt sich hier selbstständig einen Nutzernamen.
			
			\item Hauptmenü: Hier kann der Nutzer ein Offline-Spiel starten, einen anderen Spieler herausfordern oder seine Statistiken einsehen.
			
			\item Statistik Seite: Der Nutzer kann seine Spielstatistiken einsehen.
			
			\item Schachbrett: Hier kann der Nutzer in einer Partie seine Züge durchführen.
			
		\end{itemize}
	\end{itemize}
	\item[KM1050] \textbf{\textit{Schachfigur bewegen}}
	\begin{itemize}
		\item Figuren können durch antippen ausgewählt werden.
		\item Wird eine Figur angewählt, werden ihre gültigen Züge angezeigt.
		\item Es können nur gültige Züge ausgeführt werden.
		\item Die Zugberechnung wurde durch JUnit Tests überprüft, die korrekte Darstellung in mehreren Testspielen überprüft.
	\end{itemize}
\end{description}
\subsection{Wunschkriterien}
\begin{itemize}
\item{\textbf{Account/Gastzugang:}} Für das Benutzen der App wird ein Name benötigt, welcher beim ersten Öffnen eingegeben wird. 
\item{\textbf{Zwei Spieler auf einem Gerät:}} Wurde implementiert. Hierfür werden die schwarzen Figuren um 180 Grad gedreht dargestellt. \\
\\
Ansonsten wurden keine Wunschkriterien implementiert.
\end{itemize}
\section{Globale Testfälle}
\subsection{Benutzerfunktionen}
\begin{description}
	\item[T1010] \textbf{\textit{Anmelden: }} 
	\begin{description}
	\item Wurde abgewandelt in eine lokale Anmeldung und der Button Anmelden wurde nicht implementiert, da kein Gastzugang implementiert wurde.
	Beim erstmaligen Start gibt der Nutzer einen selbst gewählten Namen ein.
	\end{description}	

	\item[T1020]  \textbf{\textit{Gastzugang: }} 
    \begin{description}
	\item Nicht implementiert.
	\end{description}

	\item[T1030] \textbf{\textit{Abmelden: }}
	\begin{description}
	\item Nicht implementiert, damit der Account Gerät-gebunden ist.
	\end{description}
	
	\item[T1040] \textbf{\textit{Anzeige des eigenen, persönlichen Profils: }} 
	\begin{description}
	\item Funktioniert wie vorgesehen. 
	\end{description}
	
	\item[T1050] \textbf{\textit{Suche nach Benutzern anhand des Benutzernamens und Anzeigen der persönlichen Profile anderer Benutzer: }} 
	\begin{description}
	\item Falls der Spieler nicht vorhanden ist wird keine Meldung angezeigt, dass er nicht vorhanden ist.  Er wird gar nicht in der Liste angezeigt und man muss einen anderen vorhandenen Spieler suchen. \\
	\item Es werden alle Spieler seit Server Start aufgelistet.
	\end{description}
	
	
	\item[T1060] \textbf{\textit{Multiplayer auf einem Gerät: }} 
	\begin{description}
	\item Der Button Multiplayer wurde in Sofortspiel umbenannt und die Figuren sind dementsprechend für die Spieler umgedreht.
	Durch mehrere Testpersonen getestet.
	\end{description}
	
	
	\subsection{Initialisierung}
	\item[T2010] \textbf{\textit{Eröffnung eines Spieles: }} 
	\begin{description}
	\item Der Button Sofortspiel erzeugt ein Offline-Spiel, wo zwei Spieler gegeneinander spielen können. Die Funktion des Herausforderns eines anderen Spielers funktioniert, indem man den Gegner sucht und herausfordert.
	\end{description}
	

	
	\item[T2020] \textbf{\textit{Herausfordern und Spieleinstellungen bestimmen: }} 
	\begin{description}
	\item Gewünschte Farbe und Bedenkzeit wurden nicht implementiert.
	\item Der Nutzer der einen anderen Nutzer herausfordert, bekommt weiß zugewiesen.
	\end{description}
	

	
	\item[T2030] \textbf{\textit{Annehmen einer Herausforderung: }} 
	\begin{description}
	\item Nicht implementiert.
	\end{description}

	
	\item[T2040] \textbf{\textit{Ablehnen einer Herausforderung: }} 
	\begin{description}
	\item Nicht implementiert.
	\end{description}

	
	\subsection{Spielverlauf}
	
	\item[T3010] \textbf{\textit{Schachfiguren bewegen: }} 
	\begin{description}
	\item Funktioniert wie vorgesehen.
	\end{description}
		
	
	\item[T3020] \textbf{\textit{Remis bieten: }} 
	\begin{description}
	\item Nicht implementiert.
	\end{description}

	
	\item[T3030] \textbf{\textit{Remis annehmen: }}
	\begin{description}
	\item Nicht implementiert.
	\end{description}

	
	\item[T3040] \textbf{\textit{Remis ablehnen: }}
	\begin{description}
	\item Nicht implementiert.
	\end{description}

	
	\item[T3050] \textbf{\textit{Aufgeben: }} 
	\begin{description}
	\item Nicht implementiert.
	\end{description}

	
	\item[T3060] \textbf{\textit{Rückkampf anbieten: }}
	\begin{description}
		\item Nicht implementiert.
	\end{description}
	
	\item[T3070] \textbf{\textit{Spiel speichern: }} 
	\begin{description}
		\item Nicht implementiert.
	\end{description}
	
	
\end{description}

\section{Bugs}
\begin{itemize}
\item{\textbf{Online-Spielen:}}
	\begin{itemize}
	\item{\textbf{Symptom:}} Online-Spielen war nicht möglich.
	\item{\textbf{Ursache:}} Serverkommunikation im gleichen Thread wie die restliche App laufen zu 	lassen führte zu Fehlern. (NetworkOnMainThreadException)
	\item{\textbf{Behebung:}} Serverkommunikation mittels AsyncTask in separate Threads auslagern. 
	\end{itemize}
\item{\textbf{ Spielersuche:}} 
	\begin{itemize} 
	\item{\textbf{Symptom:}} Spielersuche nicht möglich.
	\item{\textbf{Ursache:}} Netzwerkzugriff im UI-Thread verboten. (NetworkOnMainThreadException)
	\item{\textbf{Behebung:}} Asynchrone Methode enqueue() mit Callback, statt synchroner Methode execute() verwendet. 
	\end{itemize}
\item{\textbf{Verlieren von Weiß:}}
	\begin{itemize}
	\item{\textbf{Symptom:}} Weiß konnte nicht verlieren.
	\item{\textbf{Ursache:}} Es wurde nicht abgefragt, ob nach einem Zug von schwarz das Spiel beendet war.
	\item{\textbf{Behebung:}} Die Ergebnisabfrage wurde in eine extra Methode ausgelagert, die dann an den richtigen Stellen aufgerufen wird.
	\end{itemize}
\item{\textbf{Fehlerhafte Brettübertragung:}}
	\begin{itemize}
		\item{\textbf{Symptom:}} Der Brettstring wurde korrekt in der Datenbank gespeichert, kam aber abgeschnitten in der App an.
		\item{\textbf{Ursache:}} Unbekannt.
		\item{\textbf{Behebung:}} Die Trennung der einzelnen Bereiche im Brettstring mit dem Symbol \glqq X\grqq, statt \glqq \#\grqq \space vorgenommen.
	\end{itemize}
\item{\textbf{Websocket Verbindung ohne Name:}}
\begin{itemize}
	\item{\textbf{Symptom:}} Der Nutzer meldet sich als NoUser beim Server.
	\item{\textbf{Ursache:}} Die Verbindung wurde schon aufgebaut, bevor der Nutzer einen Namen eingegeben hat.
	\item{\textbf{Behebung:}} Es wird überprüft ob der Nutzer schon einen Namen eingegeben hat.
\end{itemize}

\item{\textbf{Doppelte Namen:}}
\begin{itemize}
	\item{\textbf{Symptom:}} Kein Symptom.
	\item{\textbf{Ursache:}} Es wurde nicht geprüft ob der eingegebene Name schon vergeben ist.
	\item{\textbf{Behebung:}} Es wird überprüft ob der eingegebene Name schon vergeben ist und falls ja wird der Login mit diesem Namen unterbunden.
\end{itemize}

\item{\textbf{Herausfordern obwohl man bereits spielt:}}
\begin{itemize}
	\item{\textbf{Symptom:}} Der Nutzer bekam keine Meldung, falls er auf "Herausfordern" geklickt hat, aber bereits in einem Spiel war.
	\item{\textbf{Ursache:}} Es wurde nicht geprüft ob man selber bereits in einem Spiel ist. Der Server hat diese ungültige Abfrage aber abgefangen und kein neues Spiel erstellt. 
	\item{\textbf{Behebung:}} Der Nutzer wird nun benachrichtigt, wenn er sich bereits in einem Spiel befindet.
\end{itemize}

\end{itemize}
\end{document}