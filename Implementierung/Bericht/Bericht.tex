\documentclass[parskip=full]{scrartcl}
\usepackage[utf8]{inputenc} % use utf8 file encoding for TeX sources 

\usepackage[T1]{fontenc} % avoid garbled Unicode text in pdf 
\usepackage[german]{babel} % german hyphenation, quotes, etc 
\usepackage{hyperref} % detailed hyperlink/pdf configuration
\usepackage{graphicx}
\usepackage[toc]{glossaries}
\usepackage{caption}
\usepackage{pdfpages}
\hypersetup{ % ‘texdoc hyperref‘ for options 
	pdftitle={Entwurf}, %
	bookmarks=true,%
}
\usepackage{csquotes} % provides \enquote{} macro for "quotes"
\usepackage{enumitem}

\begin{document}
	\begin{titlepage}
		
		\centering
		\vspace*{0.2\textheight}
		{\Large Praxis der Softwareentwicklung}\\[\baselineskip]
		\vspace{2cm}
		{\Huge \textbf{Implementierungsbericht}}\\[\baselineskip]\par
		\vspace{2cm}
		{\LARGE Rukiye Devran, Tim Groß, Daniel Helmig, Orkhan Aliev}\par		
		\newpage	
		\tableofcontents
		\pagebreak
		
	\end{titlepage}
	\section{Ablauf}
	Implementiert wurde in zwei Gruppen, eine Server-Gruppe und eine App-Gruppe.
	Die App-Gruppe bestand aus Rukiye Devran, Tim Groß und Orkhan Aliev.
	Die Server-Gruppe bestand aus Daniel Helmig.
	Es wurde weitgehend unabhängig voneinander implementiert mit einigen gemeinsamen Treffen um den Fortschritt zu begutachten und eventuelle Differenzen zwischen Komponenten zu beseitigen.
	\section{Schwierigkeiten}
		\subsection{Allgemein} 
		Teilweise gab es Unstimmigkeiten zwischen den verschiedenen Komponenten Benutzeroberfläche, Schachlogik und Server, bei denen nicht ganz klar war welche Funktionalität bereit gestellt wird und was im Laufe der Implementierung geändert werden musste und somit nicht mehr zur Verfügung steht. 
		\subsection{Benutzeroberfläche}		
		\subsection{Schachlogik}
		Im Laufe der Implementierung erwiesen sich einige Klassen als überflüssig. Andere Funktionalitäten fehlten im Entwurf und mussten zusätzlich implementiert werden.
		
		
		
		
		\subsection{Server}
		Ein Hindernis war die Umstellung von Javalin mit eingebettetem Server zu einem Web-Servlet. Diese Funktionalität ist neu in Javalin und hat zu viele Probleme verursacht, sodass der gesamte ClientHandler neu geschrieben wurde.
		Dieses mal mithilfe des Spring Frameworks.	
		
		Probleme gab es zudem beim Testen des Servers aufgrund nicht optimalen Entwurfs.
		Die Klassen GameCreator, MoveHandler und BoardHandler bekamen als Parameter nur den/die Spieler beziehungsweise den Zug. Daher bekamen alle drei den DatabaseHandler direkt über die statische Methode getHandler. Dies erschwerte das Testen dieser Klassen ungemein und wurde gelöst indem der DatabaseHandler nun im Konstruktor übergeben wird.
		
		Generell erwies sich das Testen des Servers als schwierig, da sehr viele externe Bibliotheken verwendet wurden und viele statische Aufrufe gemacht werden.
		\section{Veränderungen am Entwurf}
		
		\subsection{Benutzeroberfläche}	
			
		\subsection{Schachlogik}
		
		\subsubsection{Hinzugefügte/Entfernte Klassen}
		\begin{description}
		
		\item[-Game]\hfill \\ Das Konzept eines Spiels als eigenes Objekt wurde komplett verworfen, da vom Server sowie von der GUI lediglich das Brett und der Regellieferer separat verwendet werden. 
		\item[-Player]\hfill \\ Es ist nicht nötig, Spieler zu modellieren, da das Konzept eines Spiels verworfen wurde. Spieler werden anhand ihres Namens erkannt, und so vom Server abgerufen.
		\item[-User]\hfill \\ Daten wie gewonnene/gespielte Spiele werden direkt in der App gespeichert. Über den Namen eines Spielers kann dieser direkt vom Server abgefragt werden, so kann eine eigene Modellierung umgangen werden.
		\item[+MoveFactory]\hfill \\ Das Erstellen eines Zuges anhand eines eindeutigen Strings, den man durch die toString-Methode erhält, ist an mehreren Stellen notwendig. Da es jedoch mehrere Unterklassen gibt, welche von Move erben, kann nicht einfach ein allgemeiner Konstruktor verwendet werden, welcher den String entgegen nimmt. Es ist eine eigene Klasse, eine sogennante Fabrik, notwendig, um den jeweiligen Typ von Zug anhand des Strings erstellen zu können. Die getMove-Methode der MoveFactory gibt den korrekten Typ von Zug zurück, bei fehlerhaften Argumenten wird null zurückgegeben.
		\item[+PieceFactory]\hfill \\ Wie bei Move muss auch eine Figur an mehreren Stellen anhand ihrer eindeutigen String-Repräsentation erstellt werden können. Da auch Piece eine Vererbungshierarchie aufweist, welche einen universellen Konstruktor unmöglich macht, wird auch hier eine Fabrik benötigt. Die getPiece-Methode der PieceFactory erstellt eine Figur mit ihrer korrekten Farbe und gibt diese zurück, bei falschen Argumenten wird null zurückgegeben.
		\end{description}
		
		\subsubsection{Hinzugefügte/Entfernte/Veränderte Methoden}
		\begin{description}
		\item[RuleProvider]\hfill \\ Die Methode isChecked(boolean, BoardState) wurde private gemacht, da sie nie von außerhalb benutz werden muss. \\
		\\
		Die private Methode hasLegalMoves(BoardState) wurde hinzugefügt, diese prüft ob der ziehende Spieler mögliche Züge hat. Dies wird zur Überprüfung auf Matt/Patt benötigt.\\
		\\
		Die neue private Methode isAllowedMove(Move, BoardState) prüft, ob ein Zug auf dem Brett erlaubt ist. Dazu wird nur geprüft, ob der eigene König nach dem Zug im Schach stehen würde. Diese Methode wird von getLegalMoves(Position, BoarsState) zum Filtern der von getPossibleMoves(Position, BoardState) übergebenen Züge verwendet. Die öffentliche Methode isLegalMove(Move, BoardState) prüft zusätzlich, ob an der Startposition des Zugs überhaupt die richtige Figur steht und außerdem ob der ziehende Spieler auch am Zug ist. Diese Methode wird vom Server zur Überprüfung auf Korrektheit verwendet.
		
		\item[BoardState]\hfill \\ Der Konstruktor ohne Parameter wurde entfernt. Dieser sollte ursprünglich ein Schachbrett in Startkonfiguration zurückgeben, diese Funktionalität wird aber bereits von der getStartState()-Methode des ChessRuleProviders implementiert. \\
		\\
		Es wurde die Methode hasPieceAt(Position) hinzugefügt, welche prüft ob an der gegebenen Position eine Figur steht. So werden null-Abfragen nach der Benutzung von getPieceAt(Position) vermieden, was den Code sauberer macht.\\
		\\
		Die neue Methode getPiecesOfColor(boolean) ermöglicht schnelles iterieren über alle Figuren einer bestimmten Farbe, was zum Beispiel zur Überprüfung auf Schach nützlich ist.\\
		\\
		Auch getKingOfColor(boolean) ist wichtig für die Überprüfung auf Schach. Es wird keine Fehlermeldung ausgegeben, wenn die Anzahl der Könige einer Farbe ungleich eins ist, da solche Stellungen durch das Ausführen legaler Züge nicht erreicht werden können, nur durch die gezielte Erstellung falscher Bretter mit Strings.\\
		\\
		Die Klasse Tile erhielt eine neue Methode removePiece(), welcher die Figur auf diesem Feld auf null setzt (eleganter als setPiece(null)).
		
		\item[Result]\hfill \\ Die toString()-Methode wurde in getResult() umbenannt.\\
		\\
		Der Konstruktor mit nur einem Parameter wurde aufgrund fehlender Nutzung entfernt.
		
		\item[Move]\hfill \\ Es wurde die Methode equals(Move) hinzugefügt, welche den jeweiligen Zug mit dem übergebenen vergleicht, was an vielen Stellen nützlich ist. Dazu werden einfach die Start- und Zielpositionen der Züge verglichen. Es ist nicht nötig ob es die gleichen Typen von Zug sind, da auf einem gegebenen Brett immer nur ein Zug mit gleichem Start und Ziel existiert.\\
		\\
		Die to-String()-Methode wurde bei den Spiezielzügen verändert um die Strings einheitlicher zu halten, es wird nur noch Start, Ziel und ein Zeichen als Zusatz zurückgegeben, die zusätzlichen Informationen können aus diesen errechnet werden.\\
		\\
		Die Konstruktoren von Position fangen nun fehlerhafte Eingaben ab und geben entsprechende Fehlermeldungen aus, ohne die Position zu erzeugen.\\
		\\
		Positionen haben nun eine equals(Position)-Methode, welche zwei Positionen anhand ihrer Koordinaten vergleicht.
		\item[Piece]\hfill \\ Die toString()-Methode der jeweiligen Figuren beachtet nun die Farbe der Figur, und gibt bei schwarzen Figuren den entsprechenden Kleinbuchstaben zurück. \\
		\\
		Es wurde eine Methode getImageName() für alle Figuren hinzugefügt, welche den Namen des jeweiligen Bildes zurückgibt, das diese Figur auf dem Schachbrett repräsentieren soll.
		
		
		\end{description}
		\subsection{Server}
		
\end{document}