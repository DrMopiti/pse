\documentclass[parskip=full]{scrartcl}
\usepackage[utf8]{inputenc} % use utf8 file encoding for TeX sources 

\usepackage[T1]{fontenc} % avoid garbled Unicode text in pdf 
\usepackage[german]{babel} % german hyphenation, quotes, etc 
\usepackage{hyperref} % detailed hyperlink/pdf configuration
\usepackage{graphicx}
\usepackage[toc]{glossaries}
\usepackage{caption}
\usepackage{pdfpages}
\hypersetup{ % ‘texdoc hyperref‘ for options 
	pdftitle={Entwurf}, %
	bookmarks=true,%
}
\usepackage{csquotes} % provides \enquote{} macro for "quotes"
\usepackage{enumitem}

\begin{document}
	\begin{titlepage}
		
		\centering
		\vspace*{0.2\textheight}
		{\Large Praxis der Softwareentwicklung}\\[\baselineskip]
		\vspace{2cm}
		{\Huge \textbf{Implementierungsbericht}}\\[\baselineskip]\par
		\vspace{2cm}
		{\LARGE Rukiye Devran, Tim Groß, Daniel Helmig, Orkhan Aliev}\par		
		\newpage	
		\tableofcontents
		\pagebreak
		
	\end{titlepage}
	\section{Ablauf}
	Implementiert wurde in zwei Gruppen, eine Server-Gruppe und eine App-Gruppe.
	Die App-Gruppe bestand aus Rukiye Devran, Tim Groß und Orkhan Aliev.
	Die Server-Gruppe bestand aus Daniel Helmig.
	Es wurde weitgehend unabhängig voneinander implementiert mit einigen gemeinsamen Treffen um den Fortschritt zu begutachten und eventuelle Differenzen zwischen Komponenten zu beseitigen.
	\section{Schwierigkeiten}
		\subsection{Allgemein} 
		Teilweise gab es Unstimmigkeiten zwischen den verschiedenen Komponenten Benutzeroberfläche, Schachlogik und Server, bei denen nicht ganz klar war welche Funktionalität bereit gestellt wird und was im Laufe der Implementierung geändert werden musste und somit nicht mehr zur Verfügung steht. 
		\subsection{App}
		\subsection{Server}
		Ein Hindernis war die Umstellung von Javalin mit eingebettetem Server zu einem Web-Servlet. Diese Funktionalität ist neu in Javalin und hat zu viele Probleme verursacht, sodass der gesamte ClientHandler neu geschrieben wurde.
		Dieses mal mithilfe des Spring Frameworks.	
		
		Probleme gab es zudem beim Testen des Servers aufgrund nicht optimalen Entwurfs.
		Die Klassen GameCreator, MoveHandler und BoardHandler bekamen als Parameter nur den/die Spieler beziehungsweise den Zug. Daher bekamen alle drei den DatabaseHandler direkt über die statische Methode getHandler. Dies erschwerte das Testen dieser Klassen ungemein und wurde gelöst indem der DatabaseHandler nun im Konstruktor übergeben wird.
		
		Generell erwies sich das Testen des Servers als schwierig, da sehr viele externe Bibliotheken verwendet wurden und viele statische Aufrufe gemacht werden.
		
		
\end{document}